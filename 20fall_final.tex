% !TEX program = xelatex
\documentclass[12pt]{article}
\usepackage{ctex}
\usepackage{amsmath}
\usepackage{amssymb}
\usepackage{geometry}
\usepackage{fancyhdr}
\usepackage{titlesec}
\usepackage{xcolor}
\usepackage[shortlabels]{enumitem}

\geometry{a4paper,margin=2.5cm,headheight=15pt}
\linespread{1.3}

% 设置页眉页脚
\pagestyle{fancy}
\fancyhf{}
\fancyhead[L]{Chen Gao}
\fancyhead[R]{2020年秋季学期}
\fancyfoot[C]{\thepage}

% 设置section格式
\renewcommand{\thesection}{\chinese{section}}
\titleformat{\section}
{\normalfont\large\bfseries}{第\thesection 题}{1em}{}
\titlespacing{\section}{0pt}{3.5ex plus 1ex minus .2ex}{2.3ex plus .2ex}

% 设置enumerate环境
\setenumerate[1]{label=(\alph*),leftmargin=2em}

\begin{document}

\title{产业组织理论期末考试试题}
\author{Chen Gao\thanks{北京大学国家发展研究院,仅作为本人期末复习自制答案,不保证正确性}}
\date{\today}
\maketitle

\medskip
\noindent\textbf{请回答所有问题,表述务必清楚,准确,简洁。}

\section*{第一题(16分)}
考虑一个有两个产品(1和2)的市场,其中产品1由一个垄断企业提供,产品2由完全竞争的企业提供。企业均采用线性定价,生产成本均为零。消费者是同质的,每个消费者对两个产品的需求函数分别为$q_1 = 1- p_1$和$q_2 = 1- p_2$(当价格大于1时需求量为零)。如果上述垄断企业从市场上采购产品2,然后将其与产品1进行"条件捆绑"(即消费者从垄断企业购买产品1时,必须同时从该企业购买其所需要的产品2,否则拒绝交易)后销售给消费者。请找出该垄断企业的最优条件捆绑方案($p_1^*$,$p_2^*$)。

\noindent\textbf{【提示】}如果消费者不接受垄断企业的条件捆绑方案,仍然可以在市场上单独购买产品2。

\noindent\textbf{【答案】}

在这个条件捆绑定价问题中,垄断企业需要考虑消费者的参与约束,即消费者从购买捆绑产品获得的剩余不能低于其只在竞争市场购买产品2获得的剩余。由于产品2在竞争市场的价格为零,消费者的保留剩余为$\frac{1}{2}$。

垄断企业的最优化问题可以表示为:
\[\max_{p_1,p_2} p_1(1-p_1) + p_2(1-p_2)\]
\[\text{s.t.} \quad \frac{(1-p_1)^2}{2} + \frac{(1-p_2)^2}{2} \geq \frac{1}{2}\]

使用拉格朗日乘子法求解这个约束优化问题。构造拉格朗日函数:
\[L = p_1(1-p_1) + p_2(1-p_2) + \lambda(\frac{(1-p_1)^2}{2} + \frac{(1-p_2)^2}{2} - \frac{1}{2})\]

求解一阶条件可得最优定价方案为:$p_1^* = p_2^* = \frac{3-\sqrt{5}}{2} \approx 0.2929$。在这个定价方案下,消费者剩余恰好等于保留剩余$\frac{1}{2}$,垄断企业的总利润约为0.4142。这一结果表明,通过条件捆绑,垄断企业选择对两个产品采用相同的较低价格,以满足消费者参与约束的同时实现利润最大化。

\section*{第二题(12分)}
某产品市场(反)需求函数为$P(Q)=12-Q$,有两个企业提供该产品,边际成本分别为$c_1$和$c_2$。企业之间进行无产品差异的静态产量竞争。这时两个企业的均衡产量分别为(不必证明):
\[q_1=\frac{12+c_2-2c_1}{3}\quad\text{和}\quad q_2=\frac{12+c_1-2c_2}{3}\]

企业的初始边际生产成本为$c_1=c_2=6$。现假设企业1可以通过投资K开发某项技术,该技术可以将企业1的边际成本从6降至3。请问在企业1愿意投资于该技术的前提下,该项投资能否增加社会总福利?请给出理由。

\noindent\textbf{【答案】}
让我们分别分析技术创新前后的市场均衡和社会福利。在初始状态下,两个企业的边际成本相同($c_1=c_2=6$),根据给定的均衡产量表达式,可以计算出每个企业的产量为2,总产量为4。此时市场价格为8,每个企业的利润为$(8-6)\times2=4$。消费者剩余为总产量平方的一半,即$4^2/2=8$。因此,初始状态下的社会总福利为两个企业的利润之和加上消费者剩余,即$4+4+8=16$。

当企业1采用新技术后,其边际成本降至3,而企业2的边际成本保持在6。在新的均衡中,企业1的产量增加到4,而企业2的产量减少到1,使得总产量增加到5。市场价格因此降低到7。在新均衡下,企业1的利润增加到$(7-3)\times4=16$,而企业2的利润减少到$(7-6)\times1=1$。由于总产量增加,消费者剩余也随之增加到$5^2/2=12.5$。新的社会总福利为$16+1+12.5=29.5$。

比较两种状态可知,技术创新使社会总福利增加了13.5。同时,企业1的利润增加了12,这意味着只要投资成本$K$小于12,企业1就有动力进行技术创新。而在企业1愿意投资的条件下(即$K<12$),投资必然会增加社会总福利,因为福利的增加量(13.5)大于企业1愿意投资的最大成本(12)。这一结果的经济直觉是:技术创新通过降低生产成本提高了生产效率,不仅使企业1获得了更高的利润,还通过增加市场总产量和降低价格提高了消费者福利。虽然这一创新对企业2造成了负面影响(其利润从4减少到1),但这种替代效应带来的损失远小于效率提升带来的收益,使得社会总体福利得到了显著改善。

\section*{第三题(12分)}
在某完全竞争的耐用品市场上,企业生产一件可使用L年、品质为$q\in(0,1)$的产品的成本是$c(L,q)$,该成本函数为$L$和$q$的单增函数。在该耐用品的使用期限内,性能每年衰减为上一年的$q$"倍",超出使用期限$L$后产品性能归零,但在第$L+1$年产生货币化的残值$z$。企业以出租的方式营销该产品,租金与产品性能成比例,例如若一个新品的年租价为1,那么其在第二年的租价为$q$。企业收入的跨年贴现因子为$\gamma\in(0,1)$。假设存在唯一完全竞争均衡,其中所有企业均生产耐用期限为$L^o$和品质为$q^o$的产品。请问$L^o$和$q^o$是如何决定的?

\noindent\textbf{【答案】}
在这个耐用品市场中,企业需要同时选择最优使用期限$L$和品质$q$。让我们首先分析企业的收入结构。由于租金与性能成比例,且性能每年衰减为上一年的$q$倍,企业在第$t$年($t=1,2,\ldots,L$)的租金收入为$q^{t-1}$。考虑贴现因子$\gamma$,这些租金收入的现值之和为$\sum_{t=1}^L(\gamma q)^{t-1}$。此外,在第$L+1$年还有残值$z$的现值$\gamma^Lz$。

使用几何级数求和公式,企业的收入函数可以写为:
\[R(L,q)=\frac{1-(\gamma q)^L}{1-\gamma q}+\gamma^Lz\]

企业的利润最大化问题可以表示为:
\[\max_{L,q} \frac{1-(\gamma q)^L}{1-\gamma q}+\gamma^Lz-c(L,q)\]

在完全竞争均衡下,最优的$L^o$和$q^o$由以下三个条件共同决定:

第一,关于使用期限$L$的一阶条件:
\[-\frac{(\gamma q)^L\ln(\gamma q)}{1-\gamma q}+\gamma^Lz\ln(\gamma)=\frac{\partial c}{\partial L}\]
这个条件表明,延长使用期限的边际收益(左边两项之和,包括最后一年的租金收入现值和残值的折现)应等于边际成本。

第二,关于品质$q$的一阶条件:
\[\frac{L\gamma(\gamma q)^{L-1}(1-\gamma q)+\gamma(1-(\gamma q)^L)}{(1-\gamma q)^2}=\frac{\partial c}{\partial q}\]
这个条件表明,提高品质的边际收益(左边表达式,体现在每期租金的提高上)应等于边际成本。其中分子的第一项代表品质提高对最后一期收入的影响,第二项反映了品质提高对之前各期收入的累积影响。

第三,由于市场是完全竞争的,均衡时企业的利润应为零:
\[\frac{1-(\gamma q^o)^{L^o}}{1-\gamma q^o}+\gamma^{L^o}z=c(L^o,q^o)\]

这三个条件构成了一个方程组,其解$(L^o,q^o)$就是完全竞争均衡下的最优使用期限和品质。从这些条件的具体形式可以看出:

1. 贴现因子$\gamma$通过影响未来收入的现值,对两个决策变量都有重要影响。$\gamma$越大,企业越重视未来收入,因此倾向于选择更长的使用期限和更高的品质。

2. 残值$z$主要影响使用期限的选择。较高的残值会激励企业选择较长的使用期限,因为这样可以获得更高的残值收入。

3. 成本函数的形式(特别是其导数$\frac{\partial c}{\partial L}$和$\frac{\partial c}{\partial q}$)决定了最优选择的具体数值。成本增加越快,最优使用期限和品质水平就越低。

4. 零利润条件确保了均衡的唯一性,因为它要求总收入恰好等于总成本,这反映了完全竞争市场的特征。

这些条件共同反映了企业在选择耐用品特征时面临的各种权衡:延长使用期限可以获得更多的租金收入和更高的残值,但会增加生产成本;提高品质可以增加每期的租金收入,但同样会提高成本。在完全竞争均衡中,这些权衡最终会导向一个使社会资源得到最优配置的结果。

\section*{第四题(12分)}
在一个长度为$L$的Hotelling"线形城市"的两端各有一个企业(1和2),他们以零成本提供有"垂直差异"但没有"水平差异"的产品。消费者均匀分布在城市中,他们对企业的产品有且仅有一个单位的需求,保留价格分别为(足够大的)$\nu_1$和$\nu_2$,且满足$|v_1-v_2|<3tL$。消费者在城市内的(往返)交通成本函数为$T(x)=tx$。企业之间进行静态价格竞争。这时企业的均衡价格和市场分界点分别为(不必证明):
\[p_1=\frac{\nu_1-\nu_2}{3}+tL,\quad p_2=\frac{\nu_2-\nu_1}{3}+tL,\quad\tilde{x}=\frac{\nu_1-\nu_2}{6t}+\frac{L}{2}.\]

现假设企业在选择各自价格之前可以进行广告宣传。有两种广告类型,其一是强调两个企业之间的产品差异,即企业分别通过投入广告支出$\alpha_1$或$\alpha_2$提高消费者的单位交通成本$t(\alpha_1,\alpha_2)$,其中$\frac{\partial t(\alpha_1,\alpha_2)}{\partial\alpha_1}>0$和$\frac{\partial t(\alpha_1,\alpha_2)}{\partial\alpha_2}>0$;其二是强调各自产品的价值,即企业通过投入广告支出$\beta_1$或$\beta_2$提高消费者对产品的估值$\nu_1(\beta_1)$和$\nu_2(\beta_2)$,其中$\frac{\partial v_1(\beta_1)}{\partial\beta_1}>0$和$\frac{\partial v_2(\beta_2)}{\partial\beta_2}>0$。请问这些广告是"合作型广告"(即对竞争对手有正外部性)还是"进攻型广告"(即对竞争对手有负外部性)?为什么?

\noindent\textbf{【答案】}
我们首先分析强调产品差异的$\alpha$型广告。设企业$i$的利润为其价格与需求量的乘积:
\[\pi_1 = p_1 \cdot D_1 = (\frac{v_1-v_2}{3}+tL) \cdot (\frac{v_1-v_2}{6t}+\frac{L}{2})\]
\[\pi_2 = p_2 \cdot D_2 = (\frac{v_2-v_1}{3}+tL) \cdot (L-\frac{v_1-v_2}{6t}-\frac{L}{2})\]

当企业1增加$\alpha_1$时,$t$增加。对利润函数求关于$t$的偏导数:
\[\frac{\partial \pi_1}{\partial t} = L(\frac{L}{2}+\frac{v_1-v_2}{6t})-\frac{v_1-v_2}{6t^2}(Lt+\frac{v_1-v_2}{3})\]
\[\frac{\partial \pi_2}{\partial t} = L(\frac{L}{2}-\frac{v_1-v_2}{6t})+\frac{v_1-v_2}{6t^2}(Lt-\frac{v_1-v_2}{3})\]

在给定条件$|v_1-v_2|<3tL$下,可以证明这两个偏导数均为正。这意味着当一个企业通过广告增加$t$时,不仅自身利润会上升,竞争对手的利润也会上升。其经济机制在于:$t$的增加使得消费者对价格的敏感度下降,从而削弱了价格竞争,使得两个企业都能够在保持市场份额相对稳定的情况下提高价格。因此,$\alpha$型广告具有正外部性,是一种合作型广告。

接下来分析强调产品价值的$\beta$型广告。当企业1增加$\beta_1$时,$v_1$增加。对利润函数求关于$v_1$的偏导数:
\[\frac{\partial \pi_1}{\partial v_1} = \frac{L}{6}+\frac{v_1-v_2}{18t}+\frac{Lt+\frac{v_1-v_2}{3}}{6t}\]
\[\frac{\partial \pi_2}{\partial v_1} = -\frac{L}{6}+\frac{v_1-v_2}{18t}-\frac{Lt-\frac{v_1-v_2}{3}}{6t}\]

可以看出$\frac{\partial \pi_1}{\partial v_1}>0$而$\frac{\partial \pi_2}{\partial v_1}<0$。这表明当企业1通过广告提高消费者对其产品的估值时,其自身利润会上升,但竞争对手的利润会下降。其作用机制是双重的:首先,$v_1$的提高直接导致企业1的价格上升而企业2的价格下降;其次,市场分界点向企业2方向移动($\frac{\partial \tilde{x}}{\partial v_1}=\frac{1}{6t}>0$),进一步压缩了企业2的市场份额。因此,$\beta$型广告具有负外部性,是一种进攻型广告。

综上所述,两种类型的广告具有截然不同的战略效果:强调产品差异的广告通过增加消费者的感知运输成本,削弱了价格竞争,使得两个企业都能够提高价格而不显著影响市场份额,因此对双方的利润都有利;而强调产品价值的广告通过提高消费者对自己产品的估值,不仅降低了竞争对手的价格,还抢夺了其市场份额,因此必然降低竞争对手的利润。

\section*{第五题(8分)}
大型同行企业之间的合并经常需要提前得到所在市场的反垄断机构批准。在相关反垄断实践中,反垄断机构一般把合并后企业的市场份额假定为参与合并的企业在合并前的市场份额的算术和,并在此基础上分析合并的福利影响。请问这种计算合并后企业市场份额的方法在理论上是否存在问题?为什么?

\noindent\textbf{【答案】}
这种简单的市场份额加总方法在理论上存在严重缺陷。首要问题在于它忽略了企业行为的根本性变化。合并前的市场份额是在企业相互竞争的情况下形成的,而合并后企业的行为策略会发生显著改变。例如,在古诺竞争市场中,合并后企业通常会减少产量以提高价格,因此实际的市场份额可能显著低于简单加总的结果。

其次,这种方法没有考虑市场结构的动态变化。合并不仅改变了参与企业的行为,还会引发整个市场竞争格局的变化。其他企业可能会调整自己的竞争策略来应对新的市场形势,而较高的市场价格也可能吸引新的企业进入市场。这些市场结构的动态调整都会影响最终的市场份额分布。

第三个重要问题是这种方法完全忽视了合并可能带来的效率效应。企业合并往往能够实现规模经济或范围经济,显著降低边际成本。如果效率提升足够大,合并后的企业可能会选择扩大而不是收缩产量。在这种情况下,实际的市场份额可能会高于简单加总的预期。

最后,这种方法也没有充分考虑产品差异化的影响。如果合并的企业生产高度替代的产品,简单的市场份额加总可能会低估合并的反竞争效应;反之,如果合并的企业生产互补产品,简单加总则可能高估合并后企业的市场支配力。

因此,反垄断部门在评估合并案件时,不应过分依赖这种机械的市场份额计算方法。相反,他们应该全面分析合并对企业行为策略的影响,评估可能的效率改进,考察市场进入壁垒的高低,以及预测其他市场参与者的反应。只有通过这种全面的分析,才能准确评估企业合并对市场竞争和社会福利的实际影响。

\section*{第六题(12分)}
某市场上的两个企业的边际成本均为零,他们面临的需求函数分别为
\[q_1=30-2p_1-p_2\quad\text{和}\quad q_2=30-2p_2-p_1\]
在市场博弈中,两个企业同时选择他们的价格。

a)请找出这个市场的均衡价格和企业利润;(4分)

b)假如这个两个企业合并,请找出这个市场的均衡价格和企业利润;(4分)

c)在两个企业保持独立的情况下,企业之间签订价格协议(即共同商定双方的销售价格)是否有反垄断方面的担忧?为什么?(4分)

\noindent\textbf{【答案】}

a) 在独立竞争的情况下,每个企业选择价格以最大化自身利润。企业1的利润最大化问题为:
\[\max_{p_1} \pi_1 = p_1q_1 = p_1(30-2p_1-p_2)\]
对应的一阶条件为:
\[\frac{\partial \pi_1}{\partial p_1} = 30-4p_1-p_2 = 0\]

同理,企业2的利润最大化问题为:
\[\max_{p_2} \pi_2 = p_2q_2 = p_2(30-2p_2-p_1)\]
对应的一阶条件为:
\[\frac{\partial \pi_2}{\partial p_2} = 30-4p_2-p_1 = 0\]

由对称性可知两个企业的均衡价格相等。解这个方程组得到均衡价格$p_1^* = p_2^* = 6$。此时每个企业的产量为$q_1^* = q_2^* = 30-2(6)-6 = 12$,利润为$\pi_1^* = \pi_2^* = 6 \times 12 = 72$。

b) 如果两个企业合并,合并后的企业将同时控制两个价格,其利润最大化问题为:
\[\max_{p_1,p_2} \pi = p_1q_1 + p_2q_2 = p_1(30-2p_1-p_2) + p_2(30-2p_2-p_1)\]

对$p_1$和$p_2$求导得到一阶条件:
\[\frac{\partial \pi}{\partial p_1} = 30-4p_1-2p_2 = 0\]
\[\frac{\partial \pi}{\partial p_2} = 30-4p_2-2p_1 = 0\]

由对称性可知合并后两个产品的价格相等。解这个方程组得到$p_1^m = p_2^m = 5$。此时每个产品的需求量为$q_1^m = q_2^m = 30-2(5)-5 = 15$,合并后企业的总利润为$\pi^m = 5 \times 15 + 5 \times 15 = 150$。

c) 在两个企业保持独立但签订价格协议的情况下,确实存在严重的反垄断担忧。首先,从经济理论来看,价格协议本质上是企业通过协同行为来模拟垄断效果。在本例中,如果两个企业通过协议共同设定价格,他们会选择$p_1 = p_2 = 5$,这与合并后的价格相同。这是因为价格协议和合并都使得两个企业共同选择价格来最大化总利润,因此会得到相同的结果:每个企业的需求量为15,总利润为150。虽然这个价格低于竞争均衡价格6,但这并不意味着价格协议有利于消费者福利。

其次,与企业合并不同,价格协议纯粹是反竞争的,没有任何效率提升的补偿效应。合并后的企业能够通过内部协调、资源整合等方式实现效率提升,这些潜在的效率收益可以为价格下降提供持续的动力。但价格协议不会产生这种效应,因为企业仍然是独立运营的,没有成本节约或协同效应。价格的下降仅仅是因为内部化了价格竞争的外部性,而不是源于效率的提升。

最后,从长期来看,价格协议会削弱市场的价格发现功能,降低企业创新和提高效率的动力。在本例中,如果允许价格协议,企业将失去通过提高效率来降低成本的动力,因为它们可以简单地通过协议来协调价格和产量。这种行为不仅损害了市场的动态效率,还会阻碍行业的长期发展。因此,从社会福利最大化的角度来看,这种价格协议应该受到反垄断法的严格管制。

\section*{第七题(8分)}
一个制造商通过多个零售商销售其产品,零售商之间进行有空间差异的价格竞争,同时也通过提供各种有成本的、不可被竞争对手"搭便车"的零售服务,降低消费者在购买过程中的成本(如店内搜寻成本)。试解释为什么从制造商的角度看,零售商很可能过于注重价格竞争而忽视服务的提供,从而导致过低的零售价格和过少的零售服务。

\noindent\textbf{【答案】}
零售商过度关注价格竞争而忽视服务提供的现象源于价格竞争和服务竞争的根本性差异。价格竞争具有直接性和即时性,零售商通过降价可以迅速吸引消费者并抢占市场份额,其效果容易观察和衡量。相比之下,服务投入(如产品展示、信息咨询等)需要较长时间才能显现效果,且其对销量的贡献难以准确量化,这使得零售商难以评估服务投入的回报率。

此外,价格和服务在竞争中的战略替代关系也加剧了这一问题。当一个零售商选择通过降价来竞争时,其他零售商往往被迫跟随降价以维持市场份额。这种价格竞争的压力会挤占用于服务提供的资源,因为在价格水平较低的情况下,零售商可能无力承担服务成本。即使服务不存在搭便车问题,这种价格竞争的动态过程仍然会导致服务水平的整体下降。

从制造商的角度看,这种行为模式损害了供应链的整体效率。制造商更关注长期的品牌价值和市场发展,优质的零售服务有助于提升消费者对产品的认知和满意度,从而扩大整体市场需求。然而,零售商的短期利润导向使其更倾向于通过价格竞争来争夺既有市场,而不是通过服务提升来开发新的市场需求。这种供应链中的激励不一致性需要制造商通过适当的契约安排(如最低零售价格限制)来协调。

\section*{第八题(10分)}
某"大国"有两个全球性寡头企业,企业以零成本生产同质产品,并在国内和国际市场销售。企业之间进行静态产量竞争。国内市场和国际市场的(反)需求函数均为$P(Q)=12-Q$。

a)如果该国政府可以分别对内销和出口(从量)征税或补贴,请找出最优税率$(t_1,t_2)$(负的税率即代表补贴)。(5分)

b)如果该国政府须遵守自由贸易政策(包括不对出口征税或补贴,不允许企业对国内外市场区别定价),但是可以对本国企业的产出(从量)征收商品税或进行补贴,请找出最优商品税率$t$。(5分)

【提示:政府追求本国国内的社会总福利最大化。】

\noindent\textbf{【答案】}

a) 在可以分别征税的情况下,两个企业同时选择它们在国内市场和国际市场的产量。企业1的优化问题为:
\[\max_{q_{1d},q_{1f}} \pi_1 = (12-q_{1d}-q_{2d}-t_1)q_{1d} + (12-q_{1f}-q_{2f}-t_2)q_{1f}\]

企业2的优化问题为:
\[\max_{q_{2d},q_{2f}} \pi_2 = (12-q_{1d}-q_{2d}-t_1)q_{2d} + (12-q_{1f}-q_{2f}-t_2)q_{2f}\]

企业1的一阶条件为:
\[\frac{\partial \pi_1}{\partial q_{1d}} = 12-2q_{1d}-q_{2d}-t_1 = 0\]
\[\frac{\partial \pi_1}{\partial q_{1f}} = 12-2q_{1f}-q_{2f}-t_2 = 0\]

企业2的一阶条件为:
\[\frac{\partial \pi_2}{\partial q_{2d}} = 12-2q_{2d}-q_{1d}-t_1 = 0\]
\[\frac{\partial \pi_2}{\partial q_{2f}} = 12-2q_{2f}-q_{1f}-t_2 = 0\]

由于两个企业是对称的,在古诺均衡下有$q_{1d}=q_{2d}=q_d$和$q_{1f}=q_{2f}=q_f$。将这个对称性条件代入任意一个企业的一阶条件,得到均衡产量:
\[q_d = \frac{12-t_1}{3}, \quad q_f = \frac{12-t_2}{3}\]

此时,国内市场和国际市场的总产量分别为:
\[Q_d = 2q_d = \frac{2(12-t_1)}{3}, \quad Q_f = 2q_f = \frac{2(12-t_2)}{3}\]

社会福利函数W由三部分组成。首先,消费者剩余为需求曲线下方与价格线之间的面积:
\[CS = \frac{1}{2}Q_d(12-Q_d) = \frac{1}{2}\cdot\frac{2(12-t_1)}{3}\cdot\frac{2(12-t_1)}{3} = \frac{2(t_1-12)^2}{9}\]

其次,生产者剩余为两个企业在两个市场的总利润:
\[PS = 2[(12-2q_d-t_1)q_d + (12-2q_f-t_2)q_f] = \frac{2(t_1-12)^2}{9} + \frac{2(t_2-12)^2}{9}\]

最后,政府的税收收入为对两个市场征收的总税额:
\[TR = t_1(2q_d) + t_2(2q_f) = -\frac{2t_1(t_1-12)}{3} - \frac{2t_2(t_2-12)}{3}\]

因此,社会福利函数为:
\[W = CS + PS + TR = -\frac{2t_1^2}{9} - \frac{8t_1}{3} - \frac{4t_2^2}{9} + \frac{8t_2}{3} + 96\]

对W分别关于$t_1$和$t_2$求导并令导数为零:
\[\frac{\partial W}{\partial t_1} = -\frac{4t_1}{9} - \frac{8}{3} = 0\]
\[\frac{\partial W}{\partial t_2} = -\frac{8t_2}{9} + \frac{8}{3} = 0\]

解这个方程组得到最优税率:
\[t_1^* = -6, \quad t_2^* = 3\]

这个结果表明,政府应该对内销采用负税率(即补贴),而对出口采用正税率。这是因为对内销补贴可以增加消费者剩余,而对出口征税可以获取国际市场的垄断租金。

b) 在只能采用单一税率的情况下,企业1的优化问题变为:
\[\max_{q_{1d},q_{1f}} \pi_1 = (12-q_{1d}-q_{2d}-t)q_{1d} + (12-q_{1f}-q_{2f}-t)q_{1f}\]

企业2的优化问题变为:
\[\max_{q_{2d},q_{2f}} \pi_2 = (12-q_{1d}-q_{2d}-t)q_{2d} + (12-q_{1f}-q_{2f}-t)q_{2f}\]

由对称性,在均衡时有:
\[q_{1d} = q_{2d} = q_{1f} = q_{2f} = \frac{12-t}{3}\]

此时社会福利函数为消费者剩余、生产者剩余和税收收入之和:
\[CS = \frac{2t^2}{9} - \frac{16t}{3} + 32\]
\[PS = \frac{4t^2}{9} - \frac{32t}{3} + 64\]
\[TR = -\frac{4t^2}{3} + 16t\]
\[W = CS + PS + TR = 96 - \frac{2t^2}{3}\]

将这个表达式对t求导:
\[\frac{\partial W}{\partial t} = -\frac{4t}{3}\]

解得最优税率:
\[t^* = 0\]

这个结果表明,在不能区别对待国内外市场的情况下,政府不应该采取任何税收或补贴政策。这是因为任何非零的税率都会同时影响国内和国际市场,从而降低社会总福利。

\end{document} 