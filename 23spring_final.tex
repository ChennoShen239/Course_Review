% !TEX program = xelatex
\documentclass[12pt]{article}
\usepackage{ctex}
\usepackage{amsmath}
\usepackage{amssymb}
\usepackage{geometry}
\usepackage{fancyhdr}
\usepackage{titlesec}
\usepackage{xcolor}
\usepackage[shortlabels]{enumitem}

\geometry{a4paper,margin=2.5cm,headheight=15pt}
\linespread{1.3}

% 设置页眉页脚
\pagestyle{fancy}
\fancyhf{}
\fancyhead[L]{Chen Gao}
\fancyhead[R]{2023年春季学期}
\fancyfoot[C]{\thepage}

% 设置section格式
\renewcommand{\thesection}{\chinese{section}}
\titleformat{\section}
{\normalfont\large\bfseries}{第\thesection 题}{1em}{}
\titlespacing{\section}{0pt}{3.5ex plus 1ex minus .2ex}{2.3ex plus .2ex}

% 设置enumerate环境
\setenumerate[1]{label=(\alph*),leftmargin=2em}

\begin{document}

\title{产业组织理论期末考试试题}
\author{Chen Gao\thanks{北京大学国家发展研究院,仅作为本人期末复习自制答案,不保证正确性}}
\date{\today}
\maketitle




\medskip
\noindent\textbf{请回答所有考题,表述务必清楚,准确,简洁。}

\bigskip
\section*{第一题(6分)}
考虑一个有两个产品(1和2)的市场,其中产品1由一个垄断企业提供,产品2由完全竞争的企业提供。如果上述垄断企业从市场上采购产品2,然后将其与产品1进行"条件捆绑"(即消费者从垄断企业购买产品1时,必须同时从该企业购买其所需要的产品2,否则拒绝交易)后销售给消费者。如果消费者不接受垄断企业的条件捆绑方案,仍然可以在市场上单独购买产品2,但无法买到产品1。该垄断企业的"条件捆绑"销售方式是否有利可图?请从直观上给出一个解释。

\noindent\textbf{答:}条件捆绑销售对垄断企业是有利可图的。从直观上看,原因如下:

\begin{enumerate}
\item 在不采用条件捆绑时,垄断企业只能从产品1获得垄断利润,消费者可以从竞争市场以竞争价格购买产品2。

\item 采用条件捆绑后,垄断企业可以利用消费者对产品1的支付意愿来提高产品2的销售价格。具体而言:
    \begin{itemize}
    \item 假设消费者对产品1的最大支付意愿为$v_1$,对产品2的最大支付意愿为$v_2$
    \item 竞争市场上产品2的价格为$p_2$
    \item 在不捆绑时,企业从产品1最多获得$v_1$的收入
    \item 在捆绑时,企业可以将两个产品的总价定在接近$(v_1+v_2)$的水平
    \item 只要$(v_1+v_2-p_2)>v_1$,捆绑销售就是有利可图的
    \end{itemize}

\item 这种策略之所以有效,是因为垄断企业可以利用其在产品1市场的垄断地位,将部分垄断势力延伸到产品2市场,从而获取更多利润。
\end{enumerate}

\section*{第二题(6分)}
某垄断企业提供两个产品。企业可以对两个产品分别定价销售,也可以进行比例捆绑后按一个总价销售。试给出一个数值例子,说明比例捆绑对企业而言可能是有利可图的。

\noindent\textbf{答:}考虑以下数值例子:

假设市场上有两类消费者A和B,各占50\%,他们对两种产品的支付意愿如下:
\begin{itemize}
\item A类消费者:产品1估值为100,产品2估值为20
\item B类消费者:产品1估值为20,产品2估值为100
\item 两种产品的边际成本均为0
\end{itemize}

如果分别定价:
\begin{enumerate}
\item 产品1最优定价为100,只有A类消费者购买,收入为$50\times100=5000$
\item 产品2最优定价为100,只有B类消费者购买,收入为$50\times100=5000$
\item 总收入为10000
\end{enumerate}

如果采用1:1的比例捆绑,定价为120:
\begin{enumerate}
    
\item 两类消费者的总支付意愿都是120,都会购买
\item 总收入为$100\times120=12000$
\end{enumerate}

因此,比例捆绑能带来更高的利润。这是因为捆绑销售可以利用不同消费者对不同产品的互补性偏好,减少消费者剩余,提高企业利润。

\section*{第三题(16分)}
某行业有两个寡头企业,分别记为1和2,他们的总成本函数均为$c(q)=1+9q$。市场需求函数为$P=15-Q$。现假设企业1有一个技术改造机会,能够以15的技改投资将边际成本从9降低至$c<6$的水平。

\begin{enumerate}
\item 请问企业1是否会进行该项技改投资?(8分)
\item 假如企业1进行该投资,对消费者福利有什么影响?(8分)
\end{enumerate}

\noindent\textbf{FYI}:假设市场需求函数为$P(Q)=A-Q$。当边际成本为$c$的企业为垄断者时,企业的最优产量为$Q^m=\frac{A-c}{2}$,(未减去固定成本的)毛利润为$\pi^m=\frac{(A-c)^2}{4}$;当两个寡头企业的边际成本分别为$c_1,c_2$时,企业产量竞争均衡产量分别为$q_1=\frac{A+c_2-2c_1}{3}$,$q_2=\frac{A+c_1-2c_2}{3}$,毛利润分别为

\[
\pi_1=\frac{(A+c_2-2c_1)^2}{9}\text{ 和 }\pi_2=\frac{(A+c_1-2c_2)^2}{9}。
\]

\noindent\textbf{答:}
\begin{enumerate}
\item 让我们分析企业1是否会进行技改投资:

首先计算技改前的利润。代入$A=15$,$c_1=c_2=9$到给定公式,得到$q_1=q_2=2$,$\pi_1=4$。

技改后$c_1=c$(其中$c<6$),$c_2=9$:
\begin{align*}
q_1' &= \frac{24-2c}{3} = 2(4-c)\\
q_2' &= \frac{c-3}{3}\\
\pi_1' &= \frac{4(12-c)^2}{9}
\end{align*}

技改带来的净收益为:$\frac{4(12-c)^2}{9}-19$。令此式大于0:
\[\frac{4(12-c)^2}{9}>19\]
\[(12-c)^2>\frac{171}{4}\]
\[|12-c|>\frac{3\sqrt{19}}{2}\]
\[12-c>\frac{3\sqrt{19}}{2} \text{ 或 } 12-c<-\frac{3\sqrt{19}}{2}\]
\[c<12-\frac{3\sqrt{19}}{2} \text{ 或 } c>12+\frac{3\sqrt{19}}{2}\]

由于题目要求$c<6$,且$12+\frac{3\sqrt{19}}{2}>6$,因此只要$c<12-\frac{3\sqrt{19}}{2}$,技改就是有利可图的。

\item 技改对消费者福利的影响:

技改前总产量$Q=4$,市场价格$P=11$

技改后总产量$Q'=\frac{21-c}{3}$,市场价格$P'=\frac{24+c}{3}$

消费者剩余变化:
\begin{itemize}
\item 技改前:$\frac{1}{2}(15-11)\times4=8$
\item 技改后:$\frac{(21-c)^2}{18}$
\end{itemize}

比较两者差值:
\[\frac{(21-c)^2}{18}-8=\frac{(21-c)^2-144}{18}\]

当$c<6$时,$21-c>15$,因此$(21-c)^2>225>144$,所以消费者剩余增加。这是因为技改导致总产量增加,市场价格下降。
\end{enumerate}

\section*{第四题(16分)}
某行业有一个垄断的上游企业和两个寡头下游企业,每个上游产品可用于生产一个下游产品,上游企业的边际成本均为0,两个下游企业(在购买上游产品之外)的边际成本也均为0。下游企业的产品无差异,市场需求函数为$P(Q)=A-Q$,其中参数$A$足够大。

\begin{enumerate}
\item 假设上游企业首先决定上游产品价格$w$,然后下游企业选择他们的产量。请找出这个市场的均衡总产量、价格、和产业总利润。(8分)
\item 假如上游企业可以通过合约决定下游产品的价格,请找出这个市场的均衡总产量、价格、和产业总利润。(8分)
\end{enumerate}

\noindent\textbf{答:}
\begin{enumerate}
\item 让我们用逆向归纳法求解第一种情况:

第二阶段,给定上游价格$w$,两个下游企业进行古诺竞争。每个企业的利润函数为:
\[\pi_i=(A-Q)q_i-wq_i=(A-q_1-q_2-w)q_i\]

一阶条件:
\[\frac{\partial\pi_i}{\partial q_i}=A-2q_i-q_j-w=0\]

由对称性,均衡时$q_1=q_2=q$,解得:
\[q=\frac{A-w}{3}\]

总产量$Q=2q=\frac{2(A-w)}{3}$,市场价格$P=A-Q=\frac{A+2w}{3}$

第一阶段,上游企业选择$w$最大化其利润:
\[\pi_u=w\cdot2q=\frac{2w(A-w)}{3}\]

一阶条件:
\[\frac{\partial\pi_u}{\partial w}=\frac{2(A-2w)}{3}=0\]

解得$w=\frac{A}{2}$

代入得:
\begin{itemize}
\item 总产量:$Q=\frac{A}{3}$
\item 市场价格:$P=\frac{2A}{3}$
\item 产业总利润:$\pi_{total}=\frac{A^2}{3}$
\end{itemize}

\item 如果上游企业可以通过合约决定下游产品的价格,这相当于垂直一体化的情况。

此时上游企业直接选择市场价格$P$(或等价地选择总产量$Q$)来最大化总利润:
\[\pi=P\cdot Q=(A-Q)Q\]

一阶条件:
\[\frac{\partial\pi}{\partial Q}=A-2Q=0\]

解得:
\begin{itemize}
\item 总产量:$Q=\frac{A}{2}$
\item 市场价格:$P=\frac{A}{2}$
\item 产业总利润:$\pi_{total}=\frac{A^2}{4}$
\end{itemize}

比较两种情况可以发现:
\begin{itemize}
\item 总产量:$\frac{A}{2}>\frac{A}{3}$,垂直一体化下产量更高
\item 市场价格:$\frac{A}{2}<\frac{2A}{3}$,垂直一体化下价格更低
\item 产业总利润:$\frac{A^2}{4}>\frac{A^2}{9}$,垂直一体化下总利润更高
\end{itemize}

这说明垂直一体化能够解决双重加价问题(double marginalization),提高产业效率,同时使消费者和企业都受益。
\end{enumerate}

\section*{第五题(16分)}
某耐用品垄断企业生产可使用2期的产品,边际生产成本为$c>0$。消费者在第一期对该耐用品服务的租赁需求为$R(Q)=A-Q$,第二期为$R(Q)=B-Q$,其中$R$为租价,$Q$为需求量,参数$A$和$B$满足$0<A<B$。市场仅存在2期,两期之间的贴现因子为1。企业以出租的方式获得收入。请找出垄断企业在两期的最优产量。

\noindent \textbf{答:}

由于是耐用品,企业在第一期生产$(Q_1+Q_2)$的总量,其中$Q_1$用于第一期租赁,$(Q_1+Q_2)$用于第二期租赁。考虑到第二期新增租赁量不能为负,我们需要加入约束条件$Q_2\geq0$。

垄断企业的最优化问题为:
\[\max_{Q_1,Q_2} Q_1(A-Q_1) + (Q_1+Q_2)(B-(Q_1+Q_2)) - c(Q_1+Q_2)\]
\[s.t. \quad Q_2\geq0\]

使用KKT条件求解:
\[\mathcal{L} = Q_1(A-Q_1) + (Q_1+Q_2)(B-(Q_1+Q_2)) - c(Q_1+Q_2) + \lambda Q_2\]

一阶条件:
\[\frac{\partial \mathcal{L}}{\partial Q_1} = A-2Q_1 + B-2(Q_1+Q_2) - c = 0\]
\[\frac{\partial \mathcal{L}}{\partial Q_2} = B-2(Q_1+Q_2) - c + \lambda = 0\]
\[\lambda Q_2 = 0, \quad \lambda\geq0, \quad Q_2\geq0\]

需要考虑两种情况:

(1) 内点解($Q_2>0$,因此$\lambda=0$):

此时解得:
\[Q_1^* = \frac{A}{2}, \quad Q_2^* = \frac{B-c-A}{2}\]

这个解成立的条件是$Q_2^*>0$,即$B-c>A$。

(2) 边界解($Q_2=0$):

此时解得:
\[Q_1^* = \frac{A+B-c}{4}, \quad Q_2^* = 0\]
\[\lambda^* = \frac{A-B+c}{2}\]

这个解成立的条件是$\lambda^*\geq0$,即$B-c\leq A$。

综上所述:
\begin{itemize}
\item 当$B-c>A$时,采用内点解,企业会在第二期增加租赁量。
\item 当$B-c\leq A$时,采用边界解,企业不会在第二期增加租赁量。
\end{itemize}

这个结果的经济直觉是:
\begin{enumerate}
\item 只有当第二期的需求参数$B$足够大(超过第一期需求参数$A$和边际成本$c$之和)时,企业才会在第二期增加租赁量。
\item 当第二期需求不够大时,企业会选择在第一期生产所有产品,并在两期中重复出租这些产品。
\item 边际成本$c$越高,企业越不倾向于在第二期增加租赁量,因为新增产量的成本可能超过收益。
\item 这种策略反映了企业在耐用品租赁市场中对需求差异和生产成本的权衡。
\end{enumerate}

\section*{第六题(18分)}
某行业的市场需求函数是$P=12-Q$。当前生产技术的边际成本为4。假设有一个独立实验室获得了一项技术,可以将生产的边际成本降低至2。

\begin{enumerate}
\item 如果这个市场上有一个垄断者,并且实验室可以将技术无条件授权给该垄断企业使用,请问实验室最多可以收到多少使用费?(6分)
\item 如果这个市场上有一个垄断者,并且实验室将技术以计件收费的方式授权给该垄断企业使用,请问实验室最多可以收到多少使用费?(6分)
\item 如果这个市场有两个寡头企业,他们进行产量竞争。如果实验室将技术以计件收费的方式授权给两个企业使用,请问实验室最多可以收到多少使用费?(6分)
\end{enumerate}

\noindent \textbf{答:}

\begin{enumerate}
\item 首先分析无条件授权的情况:

在技术改进前,垄断企业的利润为:
\[Q_{old} = \frac{12-4}{2} = 4, \quad P_{old} = 12-4 = 8, \quad \pi_{old} = (8-4)\times 4 = 16\]

在技术改进后,垄断企业的利润为:
\[Q_{new} = \frac{12-2}{2} = 5, \quad P_{new} = 12-5 = 7, \quad \pi_{new} = (7-2)\times 5 = 25\]

因此,技术持有者可以收取的最高使用费为:
\[\pi_{new} - \pi_{old} = 25 - 16 = 9\]

\item 对于计件收费的情况:

当技术持有者收取单位费率$r$时,垄断企业面临的实际边际成本为$(2+r)$。此时:

垄断企业的最优产量为:
\[Q(r) = \frac{12-(2+r)}{2} = \frac{10-r}{2}\]

市场价格为:
\[P(r) = 12-Q(r) = 12-\frac{10-r}{2} = \frac{14+r}{2}\]

垄断企业的利润为:
\[\pi(r) = (P(r)-(2+r))Q(r) = (\frac{14+r}{2}-(2+r))\frac{10-r}{2} = \frac{(10-r)^2}{4}\]

技术持有者的收入为:
\[R(r) = r\cdot Q(r) = r\cdot\frac{10-r}{2}\]

约束条件是垄断企业的利润不能低于原始利润16:
\[\frac{(10-r)^2}{4} \geq 16\]
\[10-r \geq 8\]
\[r \leq 2\]

因此,最优费率为$r^* = 2$,此时:
\begin{itemize}
\item 垄断企业的产量为$Q(2) = 4$
\item 技术持有者的总收入为$R(2) = 2\times4 = 8$
\end{itemize}

\item 对于两个寡头企业的情况:

两个企业同时选择产量进行竞争。给定技术使用费率$r$,每个企业$i$的最优化问题为:
\[\max_{q_i} (12-q_1-q_2-(2+r))q_i\]

一阶条件:
\begin{align*}
\text{企业1:} \quad & 12-2q_1-q_2-(2+r)=0\\
\text{企业2:} \quad & 12-q_1-2q_2-(2+r)=0
\end{align*}

由对称性,$q_1=q_2=q$,解得:
\[q(r)=\frac{12-(2+r)}{3}=\frac{10-r}{3}\]

因此:
\begin{itemize}
\item 总产量:$Q(r)=\frac{20-2r}{3}$
\item 市场价格:$P(r)=12-Q(r)=\frac{16+2r}{3}$
\item 每个企业的利润:$\pi(r)=(P(r)-(2+r))q(r)=(\frac{10-r}{3})^2$
\item 技术持有者的收入:$R(r)=r\cdot Q(r)=\frac{2r(10-r)}{3}$
\end{itemize}

约束条件是每个企业的利润不能低于原始利润约7.11(即$\frac{64}{9}$):
\[(\frac{10-r}{3})^2\geq\frac{64}{9}\]
\[|10-r|\geq 8\]
\[r\leq 2 \text{ 或 } r\geq 18\]

由于$r\geq 18$时收入必然更低,因此最优费率为$r^*=2$,此时:
\begin{itemize}
\item 每个企业的产量为$\frac{8}{3}$,总产量为$\frac{16}{3}$
\item 技术持有者的总收入为$\frac{32}{3}\approx 10.67$
\end{itemize}

比较三种情况:
\begin{itemize}
\item 无条件授权:收入为9
\item 对垄断企业计件收费:收入为8
\item 对寡头企业计件收费:收入约为10.67
\end{itemize}

这说明在寡头市场中,计件收费的方式可以为技术持有者带来最高的收益。这是因为:
\begin{enumerate}
\item 计件收费可以更好地捕获企业间竞争带来的市场扩张效应
\item 两个企业的竞争导致总产量增加,从而增加了技术使用费的总收入
\item 虽然每个企业的产量较小,但总体使用量更大
\end{enumerate}
\end{enumerate}

\section*{第七题(14分)}
某产品在全球只有两个寡头生产商,都位于某国。两个企业以零成本生产同质产品,并在国内和国际市场销售。企业之间进行静态产量竞争。假设全球消费者都是同质的,一个代表性消费者对该产品的需求为$P(Q)=12-Q$,$Q\geq0$。国际国内市场规模的比例为$\alpha:1$。假如这个两个企业合并,从而成为全球市场的垄断者。请问从本国利益出发,这个合并是否有利?

\noindent \textbf{答:}

让我们分别计算合并前后的本国福利。

(1) 合并前的情况:

在寡头竞争下,每个企业在每个市场的产量为$q=4$,市场总产量为$Q=8$,市场价格为$P=4$。

每个企业的总利润为:$\pi = 4 \times 4 \times (1+\alpha) = 16(1+\alpha)$

两个企业的总利润为:$\Pi = 32(1+\alpha)$

国内消费者剩余为:$CS_d = \frac{8^2}{2} = 32$

因此,合并前的本国福利为:$W_1 = 32(1+\alpha) + 32 = 32\alpha + 64$

(2) 合并后的情况:

在垄断情况下,每个市场的产量为$Q_m=6$,价格为$P_m=6$。

垄断企业的总利润为:$\pi_m = 6 \times 6 \times (1+\alpha) = 36(1+\alpha)$

国内消费者剩余为:$CS_m = \frac{6^2}{2} = 18$

因此,合并后的本国福利为:$W_2 = 36(1+\alpha) + 18 = 36\alpha + 54$

(3) 福利变化分析:

本国福利的变化为:$\Delta W = W_2 - W_1 = (36\alpha + 54) - (32\alpha + 64) = 4\alpha - 10$

当$\Delta W > 0$时,即$4\alpha - 10 > 0$,解得$\alpha > \frac{5}{2}$。

因此,当国际市场规模是国内市场的2.5倍以上时,合并有利于本国福利。这是因为虽然合并会降低国内消费者剩余,但当国际市场足够大时,企业从国际市场获得的额外垄断利润可以弥补国内消费者剩余的损失。

\section*{第八题(8分)}
一个国家的某产品完全依赖一个外国垄断企业,该产品的边际生产成本为常数。本国对该产品的需求函数为线性(如$Q=A-P$,其中参数$A$足够大)。在完全自由贸易条件下,本国大量进口该产品。请问在不考虑两国间贸易争端的前提下,本国政府是否有动机单方面对该产品的进口进行征税?为什么?

\noindent \textbf{答:}

本国政府有动机对进口产品征税。让我们通过分析来说明原因:

(1) 在征税前:
外国垄断企业面对需求函数$Q=A-P$,其利润最大化问题为:
\[\max_P (P-c)(A-P)\]

一阶条件得到最优价格$P^*=\frac{A+c}{2}$,此时:
\begin{itemize}
\item 产量:$Q^*=\frac{A-c}{2}$
\item 消费者剩余:$CS^*=\frac{(A-c)^2}{8}$
\end{itemize}

(2) 征收单位税率为$t$的关税后:
外国企业的利润最大化问题变为:
\[\max_P (P-c-t)(A-P)\]

一阶条件得到最优价格$P^*(t)=\frac{A+c+t}{2}$,此时:
\begin{itemize}
\item 产量:$Q^*(t)=\frac{A-c-t}{2}$
\item 消费者剩余:$CS(t)=\frac{(A-c-t)^2}{8}$
\item 关税收入:$T(t)=t\cdot Q^*(t)=t\frac{A-c-t}{2}$
\end{itemize}

(3) 本国福利为消费者剩余与关税收入之和:
\[W(t)=CS(t)+T(t)=\frac{(A-c-t)^2}{8}+t\frac{A-c-t}{2}\]

对$t$求导并令其等于0,得到最优关税:
\[t^*=\frac{A-c}{2}\]

这个关税水平是正的,且能使本国福利最大化。这说明本国确实有动机征税。

征税的经济直觉是:通过征税,本国可以从外国垄断企业手中攫取一部分垄断利润。虽然征税会提高价格,降低消费者剩余,但适当水平的关税所带来的收入可以超过消费者剩余的损失,从而提高本国总福利。这就是所谓的"最优关税"理论。

\end{document}
