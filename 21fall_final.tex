% !TEX program = xelatex
\documentclass[12pt]{article}
\usepackage{ctex}
\usepackage{amsmath}
\usepackage{amssymb}
\usepackage{geometry}
\usepackage{fancyhdr}
\usepackage{titlesec}
\usepackage{xcolor}
\usepackage[shortlabels]{enumitem}

\geometry{a4paper,margin=2.5cm,headheight=15pt}
\linespread{1.3}

% 设置页眉页脚
\pagestyle{fancy}
\fancyhf{}
\fancyhead[L]{Chen Gao}
\fancyhead[R]{2021年秋季学期}
\fancyfoot[C]{\thepage}

% 设置section格式
\renewcommand{\thesection}{\chinese{section}}
\titleformat{\section}
{\normalfont\large\bfseries}{第\thesection 题}{1em}{}
\titlespacing{\section}{0pt}{3.5ex plus 1ex minus .2ex}{2.3ex plus .2ex}

% 设置enumerate环境
\setenumerate[1]{label=(\alph*),leftmargin=2em}

\begin{document}

\title{产业组织理论期末考试试题}
\author{Chen Gao\thanks{北京大学国家发展研究院,仅作为本人期末复习自制答案,不保证正确性}}
\date{\today}
\maketitle

\medskip
\noindent\textbf{请回答所有考题,表述务必清楚,准确,简洁。}

\section*{第一题(20分)}
某垄断企业的边际成本为0,面临两种类型(记为1和2)的消费者,他们的需求函数分别为$q_1=6-p$,$q_2=8-p$。企业不能直接观察消费者类型,但是知道两种类型的消费者的比例为1:1。请找出企业的最优双两部定价方案$((T_1,p_1),(T_2,p_2))$。

\noindent\textbf{【答案】}
这是一个非线性定价问题,企业需要设计一个自我选择的菜单来区分两类消费者。让我们按照以下步骤求解:

1. 首先计算两类消费者的需求函数和消费者剩余:
   - 对于类型1:$CS_1(p) = \frac{(6-p)^2}{2}$
   - 对于类型2:$CS_2(p) = \frac{(8-p)^2}{2}$

2. 设计激励相容约束(IC)和参与约束(IR):
   - IC1: $CS_1(p_1) - T_1 \geq CS_1(p_2) - T_2$
   - IC2: $CS_2(p_2) - T_2 \geq CS_2(p_1) - T_1$
   - IR1: $CS_1(p_1) - T_1 \geq 0$
   - IR2: $CS_2(p_2) - T_2 \geq 0$

3. 根据标准理论,在最优方案下:
   - 高需求消费者(类型2)的IC约束和低需求消费者(类型1)的IR约束将是约束性的
   - 其他约束将自动满足

4. 由IC2约束:
   $\frac{(8-p_2)^2}{2} - T_2 = \frac{(8-p_1)^2}{2} - T_1$

5. 由IR1约束:
   $\frac{(6-p_1)^2}{2} = T_1$

6. 企业的利润最大化问题为:
   $\max_{p_1,p_2} \frac{1}{2}[p_1(6-p_1) + T_1] + \frac{1}{2}[p_2(8-p_2) + T_2]$

7. 将约束代入并求解,得到:
   $p_1^* = 3$
   $p_2^* = 4$
   $T_1^* = 4.5$
   $T_2^* = 8$

因此,最优的双两部定价方案为:
- 对类型1消费者:$(T_1^*,p_1^*) = (4.5,3)$
- 对类型2消费者:$(T_2^*,p_2^*) = (8,4)$

在这个方案下:
- 类型1消费者的需求量为3,总支付为$4.5 + 3 \times 3 = 13.5$
- 类型2消费者的需求量为4,总支付为$8 + 4 \times 4 = 24$
- 企业的总利润为$(13.5 + 24)/2 = 18.75$

这个方案实现了消费者的完全分离,并最大化了企业的利润。

\section*{第二题(12分)}
试从直觉上说明:

\begin{enumerate}
\item 为什么一个垄断企业一般没有动机将其产品与一个在市场上随处可得的完全竞争产品按比例捆绑后销售。(6分)

\item 为什么一个(不受政府规制的)垄断企业很可能有动机将其产品与一个完全竞争产品进行"条件捆绑"销售。(6分)
\end{enumerate}

\noindent\textbf{【答案】}

a) 垄断企业没有动机进行按比例捆绑销售的原因主要有:

1. 按比例捆绑无法增加垄断企业的市场支配力。由于竞争产品在市场上随处可得,其价格已经由市场决定,消费者可以自由选择从其他渠道购买该产品。因此,垄断企业无法通过捆绑销售来提高竞争产品的价格或限制其供给。

2. 捆绑销售可能反而降低垄断企业的利润。强制消费者按固定比例购买两种产品,限制了消费者的选择自由,可能导致一些原本愿意购买垄断产品的消费者放弃购买,从而减少垄断企业的销量和利润。

3. 捆绑会增加企业的经营成本。企业需要承担额外的采购、库存和销售成本,而这些成本无法通过捆绑销售得到补偿。

b) 垄断企业有动机进行条件捆绑销售的原因是:

1. 条件捆绑可以作为价格歧视的工具。通过要求消费者必须从垄断企业购买竞争产品,企业可以根据消费者对竞争产品的需求量来推断其支付意愿,从而实现更有效的价格歧视。

2. 可以获取竞争市场的额外利润。虽然竞争产品的单位利润很低,但通过将其与垄断产品捆绑,企业可以利用垄断产品的市场支配力来获取竞争市场的额外收益。

3. 增加了消费者的转换成本。一旦消费者接受条件捆绑,就需要同时从垄断企业购买两种产品,这增加了消费者转向其他供应商的成本,从而强化了垄断企业的市场地位。

\section*{第三题(20分)}
某行业的市场需求函数为$P(Q)=12-Q$;有两种生产技术(记为A和B)可供企业选择,分别以成本函数
$$c_A(q)=14+3q\quad\text{和}\quad c_B(q)=2+6q$$
代表。有两个企业(记为1和2)可能在这个行业经营,其中企业1拥有选择技术的先发优势。如果两个企业都进入市场,那么两个企业之间进行静态的产量竞争。

\begin{enumerate}
\item 假如企业1预期企业2不会进入市场,那么将采用何种技术?为什么?(10分)

\item 现假设博弈分两步:首先,企业1选择是否进入市场,以及进入的话采用何种技术;第二,企业2观察到企业1的选择后,决定是否进入市场,以及进入的话采用何种技术。请找出这个博弈的均衡状态(包括企业是否进入、技术选择和市场价格)(10分)
\end{enumerate}

【FYI:当边际成本为$c$的企业为垄断者时,企业的均衡(未减去固定成本的)毛利润为$\pi_m=(12-c)^2/4$。给定两个企业的边际成本$c_1,c_2$,在产量竞争均衡下,企业的毛利润分别为$\pi_1=(12+c_2-2c_1)^2/9$和$\pi_2=(12+c_1-2c_2)^2/9$】

\noindent\textbf{【答案】}

a) 如果企业1预期是垄断者,它需要比较采用两种技术时的净利润:

使用技术A时:
- 边际成本$c_A=3$
- 毛利润$\pi_A=(12-3)^2/4=20.25$
- 净利润$\pi_A^{net}=20.25-14=6.25$

使用技术B时:
- 边际成本$c_B=6$
- 毛利润$\pi_B=(12-6)^2/4=9$
- 净利润$\pi_B^{net}=9-2=7$

因此,企业1会选择技术B,因为它带来的净利润更高。虽然技术B的边际成本更高,但其固定成本显著低于技术A,在垄断市场规模有限的情况下,较低的固定成本带来的好处超过了较高边际成本的劣势。

b) 让我们用逆向归纳法求解这个两阶段博弈:

1. 第二阶段(企业2的决策):
   - 如果企业1选择技术A($c_1=3$):
     * 企业2选择A时的净利润:$(12+3-6)^2/9-14=1$
     * 企业2选择B时的净利润:$(12+3-12)^2/9-2=1$
     * 企业2不进入时的利润:0
     因此企业2会选择技术B进入市场

   - 如果企业1选择技术B($c_1=6$):
     * 企业2选择A时的净利润:$(12+6-6)^2/9-14=0$
     * 企业2选择B时的净利润:$(12+6-12)^2/9-2=0$
     * 企业2不进入时的利润:0
     因此企业2会选择不进入市场

2. 第一阶段(企业1的决策):
   - 如果选择技术A:
     * 企业2会选择技术B进入
     * 企业1的净利润:$(12+6-6)^2/9-14=0$

   - 如果选择技术B:
     * 企业2会选择不进入
     * 企业1的净利润:$7$(如a问所计算)

   - 如果不进入:
     * 利润为0

因此,均衡状态为:
- 企业1选择进入并采用技术B
- 企业2选择不进入
- 市场价格为$P=12-Q=12-(12-6)/2=9$

这个结果显示了技术选择的战略效应:企业1通过选择高边际成本的技术B,向潜在进入者传递了一个强有力的承诺,表明它会在进入后进行激烈的数量竞争,从而成功阻止了企业2的进入。

\section*{第四题(12分)}
在一个市场集中度很高的耐用机器设备市场,试解释为什么

\begin{enumerate}
\item 设备制造企业的市场力量可能很低,即企业接近于市场价格的"接受者";(6分)

\item 设备制造企业的销售量的波动幅度可能显著大于经济周期的波动幅度。(6分)
\end{enumerate}

\noindent\textbf{【答案】}

a) 设备制造企业的市场力量可能很低的原因:

1. 耐用品的跨期替代效应。由于机器设备的使用寿命长,消费者可以选择推迟购买或提前购买,这种跨期替代的可能性显著增加了需求的价格弹性。即使在当期市场集中度很高,企业也难以通过提高价格来获取超额利润,因为这会导致消费者转向二手市场或推迟购买。

2. 二手市场的竞争压力。耐用机器设备通常存在活跃的二手市场,这为新设备市场提供了有效的竞争。二手设备与新设备在功能上具有一定的替代性,这限制了制造企业提高新设备价格的能力。

3. 买方的议价能力。机器设备的购买者通常是企业,他们比个人消费者更理性,信息更充分,且购买量较大,这赋予了他们较强的议价能力。

b) 销售量波动幅度大于经济周期波动幅度的原因:

1. 更新加速器效应。当经济增长时,企业为了扩大产能需要购置新设备;而经济衰退时,不仅新增需求减少,更新需求也会推迟,导致设备需求的双重下降。这种效应使得设备需求对经济周期的反应被放大。

2. 存量调整机制。由于设备是耐用品,企业在经济好转时往往会集中补充设备存量,而在经济下行时则会推迟更新,这种存量调整行为加剧了需求的波动。

3. 预期的影响。企业对未来经济状况的预期会显著影响其设备投资决策。经济上行时的乐观预期会刺激投资,而下行时的悲观预期会抑制投资,这种预期效应进一步放大了需求波动。

\section*{第五题(6分)}
某产品市场有三个相互竞争的企业,A、B和C,每个企业都有可观的市场份额。现假设企业B和C提出合并,并报请反垄断机构批准。假如由你来审查此项合并案,而且你的目标是最大化社会总福利,请问你将如何开展工作?为什么?

\noindent\textbf{【答案】}
作为反垄断机构的审查人员,我会从以下几个方面开展工作:

1. 市场界定和竞争分析
   - 准确界定相关产品市场和地理市场的范围
   - 评估合并前后的市场集中度变化
   - 分析市场进入壁垒的高低
   - 考察剩余竞争者(企业A)的竞争能力

2. 效率效应评估
   - 研究合并可能带来的规模经济和范围经济
   - 评估技术创新和管理效率的潜在提升
   - 分析成本节约是否能够传导给消费者
   - 考察效率提升的可验证性和合并专属性

3. 协调效应分析
   - 评估合并后与企业A之间达成默契的可能性
   - 分析市场透明度和报复机制的有效性
   - 考察需求和成本条件是否有利于协调行为
   - 研究历史上的协调行为证据

4. 单边效应评估
   - 分析合并企业提价的能力和动机
   - 评估产品差异化程度
   - 考察消费者转向企业A的可能性
   - 研究企业A的产能约束

这种全面的分析方法的原因是:

1. 合并的福利效应是复杂的,需要权衡效率提升带来的收益和竞争减少带来的损失。

2. 不同市场的特征会导致相同的集中度变化产生不同的竞争效果,因此需要具体分析。

3. 合并可能带来的协调效应往往比单边效应更值得关注,特别是在剩余竞争者数量较少的情况下。

4. 效率提升的可信度和专属性的判断对于决策至关重要,需要仔细验证。

\section*{第六题(8分)}
很多消费品制造企业在开拓一个区域市场时,经常在当地选择或自行设立一个独家经销商,通过该经销商向当地的数量众多的零售企业出货,而不是直接与当地数量众多的零售企业合作。试从上游竞争和下游"搭便车"的角度分别给出一个解释。

\noindent\textbf{【答案】}

从上游竞争的角度来看:

独家经销安排可以缓解制造企业之间的竞争。如果制造企业直接与零售商交易,由于零售商数量众多且分散,很难监控和执行转售价格维持等垂直限制,容易导致零售价格战,最终损害制造企业的利润。通过独家经销商,制造企业可以更好地控制产品的分销和定价,避免过度竞争。此外,独家经销商对制造企业的依赖性较强,更有动力维护品牌价值和市场秩序。

从下游"搭便车"的角度来看:

独家经销安排可以解决零售商之间的搭便车问题。在开拓新市场时,往往需要大量的促销、广告和售后服务投入。如果直接与众多零售商合作,一些零售商可能会搭其他零售商的便车,享受他人的促销和服务投入带来的好处,而自己则通过低价竞争来吸引顾客。这种行为会挫伤零售商提供服务的积极性,损害品牌形象和长期发展。通过设立独家经销商,可以内部化这些外部性,激励经销商进行必要的市场投入。独家经销商会从整体角度考虑市场开发,合理分配资源,并通过对零售商的选择和管理来防止搭便车行为。

\section*{第七题(15分)}
某进口商品由一个外国垄断企业提供,假设其边际生产成本为0。记进口国对该产品的进口关税为t(取负值时为进口补贴)。若进口国的市场需求函数为$q=6-p$,找出使进口国社会总福利最大化的关税税率。

\noindent\textbf{【答案】}

让我们逐步分析这个最优关税问题:

1. 首先分析外国垄断企业的定价决策:
   - 企业面对的反需求函数为$p=6-q$
   - 考虑关税t后,企业的利润函数为$\pi=(6-q-t)q$
   - 一阶条件:$6-2q-t=0$
   - 解得均衡产量:$q^*(t)=(6-t)/2$
   - 均衡价格:$p^*(t)=6-q^*(t)=(6+t)/2$

2. 计算进口国的社会福利组成部分:

   a) 消费者剩余:
      $CS = \frac{1}{2}q^*(t)(6-p^*(t))$
      $= \frac{1}{2}\cdot\frac{6-t}{2}\cdot\frac{6-t}{2}$
      $= \frac{(6-t)^2}{8}$

   b) 关税收入:
      $TR = t\cdot q^*(t)$
      $= t\cdot\frac{6-t}{2}$
      $= \frac{6t-t^2}{2}$

3. 进口国的社会总福利:
   $W = CS + TR$
   $= \frac{(6-t)^2}{8} + \frac{6t-t^2}{2}$
   $= \frac{(6-t)^2}{8} + \frac{24t-4t^2}{8}$
   $= \frac{36-12t+t^2+24t-4t^2}{8}$
   $= \frac{36+12t-3t^2}{8}$

4. 求解最优关税:
   $\frac{dW}{dt} = \frac{12-6t}{8} = 0$
   解得:$t^* = 2$

5. 验证二阶条件:
   $\frac{d^2W}{dt^2} = -\frac{6}{8} < 0$
   满足最大化条件

因此,最优关税税率为$t^* = 2$。在这个税率下:
- 均衡产量:$q^*(2) = 2$
- 均衡价格:$p^*(2) = 4$
- 消费者剩余:$CS = 2$
- 关税收入:$TR = 4$
- 社会总福利:$W = 6$

这个结果的经济直觉是:通过征收适当的关税,进口国可以从外国垄断企业那里攫取一部分垄断租金。最优关税在消费者剩余损失和关税收入增加之间取得了平衡。关税太高会过度抑制进口,损害消费者福利;关税太低则无法获取足够的租金。

\section*{第八题(7分)}
某产品市场上有很多卖家和买家。产品可能有不同品质(即存在"垂直差异"),相对买家而言,卖家拥有关于产品品质的更确切的信息。给定品质,买家对产品的估值总是高于卖家。请从帕累托有效性的角度,解释在纯市场机制下这个产品市场可能出现什么问题、为什么、以及可能的解决办法。

\noindent\textbf{【答案】}

这个市场存在典型的信息不对称问题,具体而言是"柠檬市场"问题。从帕累托效率的角度来看:

1. 问题表现:
   - 高品质产品可能被挤出市场
   - 交易数量可能低于社会最优水平
   - 市场可能完全崩溃

2. 原因分析:
   - 由于买家无法识别产品品质,他们只愿意支付平均品质水平对应的价格
   - 这个价格对高品质卖家来说太低,导致他们退出市场
   - 剩余的低品质产品进一步降低买家的支付意愿
   - 这种逆向选择过程可能导致市场失灵

3. 帕累托无效性:
   - 存在潜在的互利交易无法达成
   - 高品质产品的社会价值未能实现
   - 资源配置偏离社会最优

4. 可能的解决办法:
   - 信号传递:卖家通过保修、认证等方式传递品质信息
   - 信息筛选:买家设计合约机制诱导卖家自我选择
   - 声誉机制:建立长期关系和评价系统
   - 第三方认证:引入独立的品质评估机构
   - 政府干预:制定最低品质标准和强制信息披露要求

\end{document} 