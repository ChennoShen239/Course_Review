% !TEX program = xelatex
\documentclass[12pt]{article}
\usepackage{ctex}
\usepackage{amsmath}
\usepackage{amssymb}
\usepackage{geometry}
\usepackage{fancyhdr}
\usepackage{titlesec}
\usepackage{xcolor}
\usepackage[shortlabels]{enumitem}

\geometry{a4paper,margin=2.5cm,headheight=15pt}
\linespread{1.3}



% 设置页眉页脚
\pagestyle{fancy}
\fancyhf{}
\fancyhead[L]{Chen Gao}
\fancyhead[R]{2021年秋季学期}
\fancyfoot[C]{\thepage}

% 设置section格式
\renewcommand{\thesection}{\chinese{section}}
\titleformat{\section}
{\normalfont\large\bfseries}{第\thesection 题}{1em}{}
\titlespacing{\section}{0pt}{3.5ex plus 1ex minus .2ex}{2.3ex plus .2ex}

% 设置enumerate环境
\setenumerate[1]{label=(\alph*),leftmargin=2em}

\begin{document}

\title{产业组织理论期末考试试题}
\author{Chen Gao\thanks{北京大学国家发展研究院,仅作为本人期末复习自制答案,不保证正确性}}
\date{\today}
\maketitle

\medskip
\noindent\textbf{请回答所有考题,表述务必清楚,准确,简洁。}

\section*{第一题(20分)}
某垄断企业的边际成本为0,面临两种类型(记为1和2)的消费者,他们的需求函数分别为$q_1=6-p$,$q_2=8-p$。企业不能直接观察消费者类型,但是知道两种类型的消费者的比例为1:1。请找出企业的最优双两部定价方案$((T_1,p_1),(T_2,p_2))$。

\noindent\textbf{【答案】}

企业的利润最大化问题为

$$\max\limits_{p_1,p_2,T_1,T_2}T_1+p_1(6-p_1)+T_2+p_2(8-p_2)$$

$$\begin{aligned}
\mathrm{s.t.}\quad{\frac{(6-p_{1})^{2}}{2}}-T_{1}& \geq\frac{(6-p_{2})^{2}}{2}-T_{2} \\
\frac{(8-p_{2})^{2}}{2}-T_{2}& \geq\frac{(8-p_{1})^{2}}{2}-T_{1} \\
\frac{(6-p_{1})^{2}}{2}& \geq T_{1} \\
\frac{(8-p_{2})^{2}}{2}& \geq T_{2} 
\end{aligned}$$

由于$\frac{(8-p)^2}2>\frac{(6-p)^2}2$ ,从约束条件(2)(3)中可以推出(4).又由于企业一般防止高需求消费者购买针对于低需求消费者的产品而非相反情况,我们暂时忽略条件(1).在将(2)(3)作为有效约束后,企业的最优化问题转化为

$$\max\limits_{p_1,p_2}\frac{(6-p_1)^2}{2}+p_1(6-p_1)+\frac{(8-p_2)^2}{2}-\frac{(8-p_1)^2}{2}+\frac{(6-p_1)^2}{2}+p_2(8-p_2)$$

分别对$p_{1},p_{2}$ 求导得到一阶条件,推出最优价格需要满足

$$p_1^*=2,\quad p_2^*=0.$$

代入约束条件(2)(3),可得

$$\begin{array}{c}T_1^*=8,\\\\T_2^*=22.\end{array}$$

最后,通过

$$\begin{aligned}\frac{(6-2)^2}{2}-8>\frac{(6-0)^2}{2}-22\end{aligned}$$

可以验证约束条件(1)也满足。

\section*{第二题(12分)}
试从直觉上说明:

\begin{enumerate}
\item 为什么一个垄断企业一般没有动机将其产品与一个在市场上随处可得的完全竞争产品按比例捆绑后销售。(6分)

\item 为什么一个(不受政府规制的)垄断企业很可能有动机将其产品与一个完全竞争产品进行"条件捆绑"销售。(6分)
\end{enumerate}

\noindent\textbf{【答案】}

a)垄断企业如果对其中的垄断商品按垄断价格收费而完全竞争商品按边际成本收费,并且捆绑的完全竞争商品没有超过消费者所需要的量,那么企业的利润与没有捆绑时一致;如果捆绑的完全竞争商品超过消费者所需量,反血会降低消费者对捆绑商品,进而对垄断产品的需求,导致利润减少。如果对捆绑商品的定价比上述更高,那么消费者将不会购买捆绑商品,转而只在完全竞争市场上购买完全竞争商品,垒断企业的利润减少。因此,上述的比例捆绑往往是无利可图的。

b)条件捆绑要求消费者一旦购买了垒断商品,就不能在其他企业处购买搭售的完全竞争商品。通过这种方式,垄断企业可以将自已在垄断商品上的市场势力延伸到完全竞争商品上。通过适当降低捆绑组合中垄断商品的价格并提高其中完全竞争商品的价格,消费者购买组合商品的剩余可以大于在完全竞争市场单独购买完全竞争商品的剩余,而垄断企业在完全竞争商品上回收的额外利润又可以大于在垄断商品上的利润损失。

\section*{第三题(20分)}
某行业的市场需求函数为$P(Q)=12-Q$;有两种生产技术(记为A和B)可供企业选择,分别以成本函数
$$c_A(q)=14+3q\quad\text{和}\quad c_B(q)=2+6q$$
代表。有两个企业(记为1和2)可能在这个行业经营,其中企业1拥有选择技术的先发优势。如果两个企业都进入市场,那么两个企业之间进行静态的产量竞争。

\begin{enumerate}
\item 假如企业1预期企业2不会进入市场,那么将采用何种技术?为什么?(10分)

\item 现假设博弈分两步:首先,企业1选择是否进入市场,以及进入的话采用何种技术;第二,企业2观察到企业1的选择后,决定是否进入市场,以及进入的话采用何种技术。请找出这个博弈的均衡状态(包括企业是否进入、技术选择和市场价格)(10分)
\end{enumerate}

【FYI:当边际成本为$c$的企业为垄断者时,企业的均衡(未减去固定成本的)毛利润为$\pi_m=(12-c)^2/4$。给定两个企业的边际成本$c_1,c_2$,在产量竞争均衡下,企业的毛利润分别为$\pi_1=(12+c_2-2c_1)^2/9$和$\pi_2=(12+c_1-2c_2)^2/9$】

\noindent\textbf{【答案】}

a)根据所给提示,企业1选择A,B技术的利润分别为

$$\pi_A=\frac{(12-3)^2}{4}-14=\frac{25}{4},$$

$$\pi_B=\frac{(12-6)^2}{4}-2=7.$$

比较大小可得企业1选择技术B,垄断企业倾向于采用固定成本低而边际成本高的技术,因为它可以通过稍提高价格把高的边际成本部分转嫁给消费者。

b)

\begin{center}
\begin{tabular}{c|ccc}
 & $A$ & $B$ & $N$ \\
\hline
$A$ & $-5,-5$ & $2,-1$ & $\frac{25}{4},0$ \\
$B$ & $-1,2$ & $2,2$ & $7,0$ \\
$N$ & $0,\frac{25}{4}$ & $0,7$ & $0,0$ \\
\end{tabular}
\end{center}

以行代表企业1的行为(A,B,N分别表示选择技术A,选择技术B,不进入),列代表企业2的行为,两者利润如表中所示。

观察可得,(A,NAB)和(A,NBB)都是子博弈精炼纳什均衡,均衡结果最终为企业1选择技术A,企业2不进入。此时$p^*=\frac{15}2$,$q^*=\frac92$。通过选择固定成本高而边际成本低的技术,先行企业用可置信的威胁达到了阻碍进入的目的。

\section*{第四题(12分)}
在一个市场集中度很高的耐用机器设备市场,试解释为什么

\begin{enumerate}
\item 设备制造企业的市场力量可能很低,即企业接近于市场价格的"接受者";(6分)

\item 设备制造企业的销售量的波动幅度可能显著大于经济周期的波动幅度。(6分)
\end{enumerate}

\noindent\textbf{【答案】}

a)耐用机器设备很可能存在二手市场,由于新产品与二手产品之间存在较大的替代性,虽然机器设备市场在当期集中度较高,面临较少的市场竞争,但二手产品将成为当期产品的竞争者,压缩当期产品的市场份额,消费者对当期产品的弹性增大,从而机器设备厂商的市场势力降低,其价格更接近完全竞争价格。

b)在经济下行时,对耐用品的总体需求下降,由于存在大量的二手产品仍可使用,对新产品的需求往往下降幅度更大,在经济上行阶段也是类似。因此耐用品销售量的波动幅度会大于经济周期本身的波动幅度。

\section*{第五题(6分)}
某产品市场有三个相互竞争的企业,A、B和C,每个企业都有可观的市场份额。现假设企业B和C提出合并,并报请反垄断机构批准。假如由你来审查此项合并案,而且你的目标是最大化社会总福利,请问你将如何开展工作?为什么?

\noindent\textbf{【答案】}

虽然合并会导致市场集中度的增加,往往使价格上升,损害消费者福利,但仅凭这一个方面并不能对社会福利作出整体断言,合并也可能有诸多好处。

合并可能带来企业的成本节约,如果成本下降带来的福利增长超过市场势力上升导致提价的福利损失,合并可能被批准。

如果市场上存在产品差异,部分企业的合并可能会对未合并企业产生正外部性,导致市场上所有企业的利润都上升。

合并还有可能使企业能够控制更多稀缺生产要素,形成规模经济效应,能够以更低的成本生产更高的产量。

因此,除了市场集中度,还至少需要考虑以上几种因素。

\section*{第六题(8分)}
很多消费品制造企业在开拓一个区域市场时,经常在当地选择或自行设立一个独家经销商,通过该经销商向当地的数量众多的零售企业出货,而不是直接与当地数量众多的零售企业合作。试从上游竞争和下游"搭便车"的角度分别给出一个解释。

\noindent\textbf{【答案】}

从上游竞争的角度,独家的经销商更容易形成双重寡头的市场结构,使得上游品牌间的相互竞争减弱,均衡的价格将更接近垒断价格,从而提升上游企业的收益。

丛下游搭便车的角度,如果存在多个下游经销商,一家经销商提供的销售服务往往会被其他经销商搭便车,因此销售服务提供往往不足。独家经销商解决了搭便车问题,使下游经销商愿意提供更多销售服务,让消费者了解产品,产生事高的购买意愿,有利干提升上游利润。

\section*{第七题(15分)}
某进口商品由一个外国垄断企业提供,假设其边际生产成本为0。记进口国对该产品的进口关税为t(取负值时为进口补贴)。若进口国的市场需求函数为$q=6-p$,找出使进口国社会总福利最大化的关税税率。

\noindent\textbf{【答案】}

该外国垄断企业的利润函数为

$$\pi=(p-t)(6-p),$$

利润最大化问题的一阶条件给出企业的最优选择为

$$p^*=\frac{6+t}{2},\quad q^*=\frac{6-t}{2}.$$

考虑到企业的定价决策后,政府选择关税水平以最大化本国社会福利,即消费者剩余与关税收人之和

$$V=\int_{0}^{q^{*}(t)}p(q)-p^{*}(t)dq+tq^{*}(t)=\int_{0}^{\frac{6-t}{2}}6-q-\frac{6+t}{2}dq+\frac{t(6-t)}{2}=\frac{(6+3t)(6-t)}{8}.$$

一阶条件给出关税的最优选择为

$$t^{*}=2.$$

\section*{第八题(7分)}
某产品市场上有很多卖家和买家。产品可能有不同品质(即存在"垂直差异"),相对买家而言,卖家拥有关于产品品质的更确切的信息。给定品质,买家对产品的估值总是高于卖家。请从帕累托有效性的角度,解释在纯市场机制下这个产品市场可能出现什么问题、为什么、以及可能的解决办法。

\noindent\textbf{【答案】}

可能出现逆向选择问题。因为当消费者对产品的信息了解程度不如卖家时,其无法确定所面对产品的品质,往往不愿意支付很高的价格。这样的情况下,拥有产品质量最高的卖家也不愿意在不够高的价格下卖出他们的产品,最终市场上成交的只有品质较低的产品,面从帕累托有效性角度也应该成交的高品质产品却没有成交。

解决方案有以下几类。第三方机构可以对交易涉及的产品进行调查,为消费者提供信息服务。消费者可以采用甄别机制,如要求卖家提供质量保证或保修合同,愿意提供的往往是产品品质高的卖家。卖家可以传递信号,如高品质产品的卖家主动提供保修服务来使自已区别于其他卖家。形成重复博弈,卖家通过多期的交易建立起自己的声誉,使消费者了解自已的产品品质。政府通过补贴等方式干预促成更多交易。

\end{document} 