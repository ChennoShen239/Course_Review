% !TEX program = xelatex
\documentclass[12pt]{article}
\usepackage{ctex}
\usepackage{amsmath}
\usepackage{amssymb}
\usepackage{geometry}
\usepackage{fancyhdr}
\usepackage{titlesec}
\usepackage{xcolor}
\usepackage[shortlabels]{enumitem}

\geometry{a4paper,margin=2.5cm,headheight=15pt}
\linespread{1.3}

% 设置页眉页脚
\pagestyle{fancy}
\fancyhf{}
\fancyhead[L]{Chen Gao}
\fancyhead[R]{2023年秋季学期}
\fancyfoot[C]{\thepage}

% 设置section格式
\renewcommand{\thesection}{\chinese{section}}
\titleformat{\section}
{\normalfont\large\bfseries}{第\thesection 题}{1em}{}
\titlespacing{\section}{0pt}{3.5ex plus 1ex minus .2ex}{2.3ex plus .2ex}

% 设置enumerate环境
\setenumerate[1]{label=(\alph*),leftmargin=2em}

\begin{document}

\title{产业组织理论期末考试试题}
\author{Chen Gao\thanks{北京大学国家发展研究院,仅作为本人期末复习自制答案,不保证正确性}}
\date{\today}
\maketitle

\medskip
\noindent\textbf{请回答所有考题,表述务必清楚,准确,简洁。}

\section*{第一题(3分)}
一个企业以超低价格销售产品,试图造成竞争对手严重亏损并被迫退出市场,从而获得市场垄断地位,这种策略被称为"掠夺性定价"。在成熟经济体,掠夺性定价大多违反反垄断法。

\begin{enumerate}
\item 很多经济学家认为,掠夺性定价一般来说很难有利可图,因此反垄断机构不应轻易指控企业。
\item 一个企业先进行低价销售,当竞争对手退出市场后,再提高价格。请问这种行为是否一定是掠夺性定价?为什么?
\end{enumerate}

\noindent\textbf{答:}

(a) 经济学家认为掠夺性定价通常难以获利,主要是因为这种策略需要企业承担巨大的当期损失。企业必须长期以低于成本的价格销售,且市场份额越大,损失就越大,这需要企业具有强大的财务实力。同时,策略的成功也充满不确定性:竞争对手可能有足够的资金实力承受价格战,即使现有竞争对手退出,也可能有新企业进入市场。更重要的是,即使策略成功,后期通过提价获得的垄断利润现值也可能无法弥补前期的损失,因为高价格可能导致消费者减少购买或寻找替代品,同时也可能吸引新的竞争者进入市场。

(b) 低价销售后提价的行为并不一定构成掠夺性定价。价格变动可能反映成本或需求的变化,也可能是正常的促销活动或新产品的"渗透定价"策略。判断掠夺性定价需要考察价格是否低于成本、是否具有排除竞争的明确意图、企业是否有能力在竞争对手退出后通过提价收回损失,以及市场是否存在足够高的进入壁垒。此外,还需要综合考虑市场结构、企业的成本结构和财务状况、价格变动的持续时间和幅度,以及对消费者福利的长期影响。因此,仅凭价格先降后升的现象,不足以认定存在掠夺性定价行为,需要进行全面的经济分析才能做出判断。

\section*{第二题(12分)}
对某产品的(反)需求函数为$P=22-Q$。有两个企业(A和B)生产该产品,没有潜在进入者。企业A采用资本密集型技术,边际成本为$w+3k$,企业B采用劳动力密集型技术,边际成本为$3w+k$,其中$w$为行业工资水平,$k$为资金价格。在初始状态下,工资水平和资金价格为$w=k=1$,两个企业的固定成本均为零。假设企业可以通过游说政府或工会组织,将行业工资水平从1提高到2,请问企业A是否有动机推动该议题?为什么?

\noindent\textbf{FYI:}当市场反需求函数为$P(Q)=A-Q$,两个寡头企业的边际成本分别为$c_1,c_2$时,产量竞争均衡产量分别为
\[q_1=\frac{A+c_2-2c_1}{3},\quad q_2=\frac{A+c_1-2c_2}{3}。\]

\noindent\textbf{答:}让我们通过计算分析企业A是否有动机推动工资上涨:

1. 初始状态($w=k=1$):
\begin{itemize}
\item 企业A的边际成本:$c_1=1+3\cdot1=4$
\item 企业B的边际成本:$c_2=3\cdot1+1=4$
\item 代入$A=22$,得到均衡产量:
\[q_1=q_2=\frac{22+4-2\cdot4}{3}=6\]
\item 市场价格:$P=22-12=10$
\item 企业A的利润:$\pi_1=(10-4)\cdot6=36$
\end{itemize}

2. 工资上升后($w=2,k=1$):
\begin{itemize}
\item 企业A的边际成本:$c_1=2+3\cdot1=5$
\item 企业B的边际成本:$c_2=3\cdot2+1=7$
\item 均衡产量:
\[q_1=\frac{22+7-2\cdot5}{3}=\frac{19}{3}\approx6.33\]
\[q_2=\frac{22+5-2\cdot7}{3}=\frac{13}{3}\approx4.33\]
\item 市场价格:$P=22-(19/3+13/3)=22-32/3\approx11.33$
\item 企业A的利润:$\pi_1=(11.33-5)\cdot6.33\approx40.11$
\end{itemize}

3. 分析结论:

企业A有动机推动工资上涨,因为:
\begin{itemize}
\item 利润变化:$\Delta\pi_1=40.11-36=4.11>0$
\item 工资上涨对企业A的影响较小(成本增加1),但对企业B的影响较大(成本增加3)
\item 这导致企业B减少产量,市场价格上升,企业A可以扩大市场份额
\item 价格上涨带来的收益超过了成本上升的损失
\end{itemize}

这个例子说明,在寡头竞争中,企业可能会策略性地支持提高行业成本,只要这种成本上升对竞争对手的影响更大。这种"提高竞争对手成本"的策略(raising rivals' costs)可以帮助企业获得竞争优势。

\section*{第三题(4分)}
试解释为什么一手(新)耐用机器设备的销售量往往对经济景气非常敏感。

\noindent\textbf{答:}一手(新)耐用机器设备的销售量对经济景气高度敏感,主要有以下几个原因:

\begin{enumerate}
\item 投资决策的可延迟性:
    \begin{itemize}
    \item 企业可以推迟购买新设备,继续使用现有设备
    \item 在经济不景气时,企业倾向于延迟设备更新,等待经济形势好转
    \item 这种可选择性使得设备购买决策对经济预期特别敏感
    \end{itemize}

\item 二手市场的存在:
    \begin{itemize}
    \item 经济不景气时,二手设备供给增加,价格下降
    \item 这为企业提供了替代新设备的选择
    \item 降低了企业购买新设备的意愿
    \end{itemize}

\item 产能利用率的影响:
    \begin{itemize}
    \item 经济不景气时,现有设备的产能利用率下降
    \item 降低了企业扩大产能的需求
    \item 减少了对新设备的购买需求
    \end{itemize}

\item 融资约束:
    \begin{itemize}
    \item 经济不景气时,企业面临更严格的融资约束
    \item 设备投资通常需要大额资金
    \item 融资困难会显著影响新设备的购买决策
    \end{itemize}
\end{enumerate}

这些因素的综合作用导致新设备销售量会随经济景气程度发生较大波动。

\section*{第四题(18分)}
某垄断企业生产可使用2期的产品,边际生产成本为6,固定成本为零。消费者在第一期对该耐用产品所提供的服务的反需求函数为$R(Q)=12-Q$,在第二期的反需求函数为$R(Q)=24-Q$,其中$R$代表租价,$Q$代表当期消费量。假设这个市场仅存在2期,且两期之间的贴现因子为1,因此企业最大化两期利润之和。

\begin{enumerate}
\item 假设企业通过出租的方式获得收入,策略变量是在两期的出租量。请找出企业在两期最优的生产和出租量。(8分)
\item 假设企业只能通过出售产品获得收入,且不能可信地承诺未来的销售量,策略变量是在两期的销售量。请找出在"子博弈完美均衡"下,企业在两期最优的销售量。(10分)
\end{enumerate}

\noindent\textbf{答:}

(a) 在租赁模式下:

企业的最优化问题为:
\[\max_{Q_1,Q_2} Q_1(12-Q_1) + (Q_1+Q_2)(24-(Q_1+Q_2)) - 6(Q_1+Q_2)\]

其中$Q_1$为第一期租赁量,$Q_2$为第二期新增租赁量。

一阶条件:
\[\frac{\partial \pi}{\partial Q_1} = 12-2Q_1 + 24-2(Q_1+Q_2) - 6 = 0\]
\[\frac{\partial \pi}{\partial Q_2} = 24-2(Q_1+Q_2) - 6 = 0\]

解得:
\begin{itemize}
\item 第一期租赁量:$Q_1^* = 3$
\item 第二期新增租赁量:$Q_2^* = 6$
\item 第二期总租赁量:$Q_1^* + Q_2^* = 9$
\end{itemize}

(b) 在销售模式下,需要使用逆向归纳法求解:

第二期问题:
给定第一期销量$Q_1$,第二期消费者的需求函数为$P_2 = 24 - (Q_1 + q_2)$,
其中$q_2$为第二期新增销量。企业的最优化问题为:
\[\max_{q_2} (24-(Q_1+q_2)-6)q_2\]

解得第二期最优新增销量:
\[q_2^*(Q_1) = \frac{18-Q_1}{2}\]

第一期问题:
消费者预期第二期价格,因此第一期愿意支付的价格为:
\[P_1 = 12-Q_1 + (24-(Q_1+q_2^*(Q_1)))\]

企业的最优化问题为:
\[\max_{Q_1} (P_1-6)Q_1 + (24-(Q_1+q_2^*(Q_1))-6)q_2^*(Q_1)\]

解得:
\begin{itemize}
\item 第一期销量:$Q_1^* = 2$
\item 第二期新增销量:$q_2^* = 8$
\item 第二期总销量:$Q_1^* + q_2^* = 10$
\end{itemize}

比较两种模式:
\begin{enumerate}
\item 租赁模式下,企业可以完全控制每期的市场供给量,避免了时间不一致性问题
\item 销售模式下,由于企业无法承诺第二期的销量,消费者会预期到第二期价格下降,这降低了第一期的需求
\item 销售模式的总销量(10)大于租赁模式的最大租赁量(9),这反映了耐用品垄断企业面临的承诺问题
\item 租赁模式能够帮助企业获得更高的利润,这是因为租赁避免了二手市场的竞争,使企业能够更好地进行跨期价格歧视
\end{enumerate}

\section*{第五题(14分)}
某旧车市场上有相同数量的卖家和买家,他们均为"价格接受者"。每个卖家有一辆旧车准备出售,每个买家需要购买一辆旧车。卖家提供不同品质的旧车,品质$v$服从区间$[0,A]$上的均匀分布,卖家的保留价格(即最低愿意出售的价格)为$kv$,其中参数$k\in(0,1)$。买家同质且为风险中性,他们对品质为$v$的旧车最高愿意支付的价格为$v$。品质$v$是卖家的私人信息,买家在购买之前无法观察,购买完成之后也无法要求卖家承担任何责任。请找出这个市场的均衡价格和实现交易的旧车品质区间。

\noindent\textbf{答:}



1. 高价格区间分析:
\begin{itemize}
\item 当$p/k\geq A$时,所有卖家都愿意出售,此时:
    \begin{itemize}
    \item 买家期望品质为$\frac{1}{2}A$(均匀分布的均值)
    \item 买家愿意支付的价格为$\frac{1}{2}A$
    \item 要使所有卖家愿意出售,需要$p\geq kA$
    \item 因此需要$kA\leq\frac{1}{2}A$,即$k\leq\frac{1}{2}$
    \end{itemize}
\end{itemize}

2. 低价格区间分析:
\begin{itemize}
\item 当$p/k<A$时,品质在$[0,p/k]$区间的卖家愿意出售,此时:
    \begin{itemize}
    \item 买家期望品质为$\frac{p}{2k}$
    \item 买家愿意支付的价格为$\frac{p}{2k}$
    \item 均衡要求$p=\frac{p}{2k}$,解得$p=0$
    \end{itemize}
\end{itemize}

\noindent\textbf{结论}

1. 当$k\leq\frac{1}{2}$时:
\begin{itemize}
\item 存在两类均衡:
    \begin{itemize}
    \item 高价格均衡:$p^*\in[kA,\frac{1}{2}A]$,交易品质区间为$[0,A]$
    \item 低价格均衡:$p^*=0$,交易品质为$v=0$
    \end{itemize}
\end{itemize}

2. 当$k>\frac{1}{2}$时:
\begin{itemize}
\item 仅存在一个均衡:
    \begin{itemize}
    \item 低价格均衡:$p^*=0$,交易品质为$v=0$
    \end{itemize}
\end{itemize}

这个结果表明,市场均衡的性质强烈依赖于卖家的保留价格系数$k$。当$k$较小时,市场可能维持在一个有效的高价格均衡区间内;当$k$较大时,市场必然失灵。这种结果反映了信息不对称对市场效率的重要影响。特别地,由于买卖双方数量相同,均衡必须是全部交易或全不交易,这进一步加剧了市场失灵的可能性。

\section*{第六题(6分)}
上游制造商和下游经销商之间经常存在显性或隐性的"转售价格维持"合约,在实践中,这类合约大多限定下游经销商的最低转售价格。试解释:

\begin{enumerate}
\item 为什么"转售价格维持"现象看起来是一个"谜"?(3分)
\item Telser(1960)认为,当"复杂"产品的销售需要下游经销商提供售前服务时,转售价格维持可能提高行业效率。请说明其中的逻辑。(3分)
\end{enumerate}

\noindent\textbf{答:}

(a) 转售价格维持之所以看起来是一个"谜",是因为这种行为似乎违背了传统经济学理论对制造商行为的预期。从理论上看,上游制造商应该希望下游经销商之间充分竞争,因为这样可以降低零售价格,扩大销量,同时制造商可以通过批发价格获取利润。下游竞争还有助于消除双重加价问题。然而在现实中,制造商却经常通过设定最低转售价格来限制下游竞争,维持较高的零售价格,这种行为看似减少了自己的利润。这种表面上违背制造商自身利益的行为,构成了一个需要解释的"谜"。

(b) Telser的特殊服务论为这个看似矛盾的现象提供了一个合理解释。对于复杂产品而言,售前服务对销售至关重要,这包括产品展示、技术咨询和使用培训等。如果没有价格限制,市场上就会出现严重的"搭便车"问题:一些经销商投入资源提供完整的售前服务,而其他经销商则不提供服务但以低价销售。消费者可能在提供服务的经销商处获取信息和体验,最终却在低价经销商处完成购买。这种行为会导致提供服务的经销商无法收回其服务成本。

转售价格维持通过保证经销商的利润率来解决这个问题。当所有经销商都必须维持一定的价格水平时,他们就有动力和资源去提供必要的售前服务,因为不再需要担心其他经销商的价格竞争。这种机制防止了低价经销商的搭便车行为,从而提高了整个分销体系的效率。因此,转售价格维持可以被视为一种解决市场失灵的制度安排,通过限制价格竞争来维持服务竞争,最终提高行业效率和消费者福利。

\section*{第七题(4分)}
Warren-Boulton(1974)认为,如果一个市场中的某个上游企业具有市场垄断力量,而下游市场高度竞争,那么该垄断企业通过向下游进行垂直整合,可以将其垄断力量延伸至下游市场,并且这么做可能有利可图。试从这个理论的角度解释为什么一些具有市场垄断力量的设备制造商倾向于向下游企业出租设备(从而可以决定设备的维修和维护方式),而不是将设备所有权出售给下游企业。

\noindent\textbf{答:}从Warren-Boulton的理论角度来看,设备制造商选择出租而非出售设备的行为可以被理解为一种垂直整合的策略。设备的维修维护构成了一个重要的下游市场,这个市场通常是竞争性的,有多个服务提供商,而且维修维护服务的需求与设备使用密切相关。

通过出租设备,制造商实际上实现了向下游市场的垂直整合。出租使制造商保留了设备的所有权,并可以在租赁合同中规定维修维护条款,从而将设备销售和服务捆绑在一起。这种安排使制造商能够利用其在设备市场的垄断地位来控制维修维护服务的提供方式,从而将垄断力量延伸到下游的服务市场。

这种策略带来了多重收益:制造商不仅获得了维修维护市场的额外收入,还能够控制设备的使用质量和标准,建立长期的客户关系,并获取设备使用的信息和反馈。通过这种出租实现的垂直整合策略,制造商成功地将设备市场的垄断力量延伸到了维修维护市场,从而获得了更高的总体利润。

\section*{第八题(12分)}
某行业中的两个独立企业生产完全同质的产品,由企业各自的生产技术决定的成本函数分别为
\[c_1(q)=\begin{cases}2+q^2 & \text{if }q>0 \\ 0 & \text{if }q=0\end{cases}\quad\text{和}\quad c_2(q)=\begin{cases}1+2q^2 & \text{if }q>0 \\ 0 & \text{if }q=0\end{cases}\]

假设这两个企业进行水平合并,请找出合并后的企业的成本函数。

\noindent\textbf{答:}让我们分析合并后企业的成本函数。对于任意给定的总产量$Q>0$,需要考虑三种可能的生产方案:

1. 只使用企业1生产:
\begin{itemize}
\item 成本函数为:$c_1(Q)=2+Q^2$
\end{itemize}

2. 只使用企业2生产:
\begin{itemize}
\item 成本函数为:$c_2(Q)=1+2Q^2$
\end{itemize}

3. 两个企业都生产:
\begin{itemize}
\item 最优化问题:
\[\min_{q_1,q_2} (2+q_1^2)+(1+2q_2^2)\]
\[s.t. \quad q_1+q_2=Q, \quad q_1,q_2>0\]
\item 构建拉格朗日函数:
\[\mathcal{L}=2+q_1^2+1+2q_2^2+\lambda(Q-q_1-q_2)\]
\item 一阶条件:
\[\frac{\partial \mathcal{L}}{\partial q_1}=2q_1-\lambda=0\]
\[\frac{\partial \mathcal{L}}{\partial q_2}=4q_2-\lambda=0\]
\[\frac{\partial \mathcal{L}}{\partial \lambda}=Q-q_1-q_2=0\]
\item 解得:
\[q_1=\frac{2Q}{3}, \quad q_2=\frac{Q}{3}\]
\item 此时总成本为:
\[c_3(Q)=3+(\frac{2Q}{3})^2+2(\frac{Q}{3})^2=3+\frac{2Q^2}{3}\]
\end{itemize}

通过比较这三个成本函数在不同区间的大小,我们可以找到最优的生产方案。具体来说:

1. $c_1$和$c_2$的交点:$Q=1$
2. $c_2$和$c_3$的交点:$Q\approx1.22$($Q=\sqrt{\frac{3}{2}}$)
3. $c_1$和$c_3$的交点:$Q\approx1.73$($Q=\sqrt{3}$)

比较不同区间的函数值可知,合并后的成本函数为:
\[c(Q)=\begin{cases}
0 & \text{if }Q=0 \\
1+2Q^2 & \text{if }0<Q\leq1 \\
2+Q^2 & \text{if }1<Q\leq\sqrt{\frac{3}{2}} \\
3+\frac{2Q^2}{3} & \text{if }Q>\sqrt{\frac{3}{2}}
\end{cases}\]

这个成本函数反映了合并企业如何根据生产规模选择最优的生产方案:
\begin{itemize}
\item 当产量较小($Q\leq1$)时,只使用固定成本较低的企业2生产
\item 当产量适中($1<Q\leq\sqrt{\frac{3}{2}}$)时,只使用边际成本增长较慢的企业1生产
\item 当产量较大($Q>\sqrt{\frac{3}{2}}$)时,两个企业共同生产以分散边际成本的快速上升
\end{itemize}

这种生产安排充分利用了两个企业的比较优势,实现了成本最小化。

\section*{第九题(14分)}
某国有一个企业是某产品在全球市场的垄断供应商,该企业的生产成本为零。对该产品的国内需求为$Q(P)=A-P$,国外需求为$Q(P)=B-P$,其中$A>0$和$B>0$为外生给定的参数,满足$A<3B$和$B<3A$。根据国际贸易规则,该企业必须对国内国外市场统一定价,但是该国政府可以对该产品的产出从量征税或补贴。假设该国政府试图通过征税或补贴实现本国社会福利最大化。请找出该国政府的最优税率$t$(负的税率即为补贴)。

\noindent\textbf{FYI:}本国社会福利包括本国企业利润、本国消费者福利和本国政府收入。

\noindent\textbf{答:}让我们分步分析这个问题:

1. 垄断企业的定价决策:
\begin{itemize}
\item 总需求函数:$Q_{total}=(A-P)+(B-P)=(A+B)-2P$
\item 企业面临单位税率$t$,其利润函数为:
\[\pi=(P-t)(A-P)+(P)(B-P)\]
\item 一阶条件:
\[\frac{\partial \pi}{\partial P}=(A-2P+t)+(B-2P)=0\]
\item 解得最优价格:
\[P^*(t)=\frac{A+B+t}{4}\]
\end{itemize}

2. 均衡数量:
\begin{itemize}
\item 国内销量:$Q_d(t)=A-P^*(t)=\frac{3A-B-t}{4}$
\item 国外销量:$Q_f(t)=B-P^*(t)=\frac{-A+3B-t}{4}$
\end{itemize}

3. 社会福利分析:
\begin{itemize}
\item 企业利润:
\[\pi(t)=(P^*(t)-t)Q_d(t)+P^*(t)Q_f(t)\]
\item 消费者剩余:
\[CS(t)=\frac{1}{2}(Q_d(t))^2\]
\item 税收收入:
\[T(t)=t\cdot Q_d(t)\]
\item 总社会福利:
\[W(t)=\pi(t)+CS(t)+T(t)\]
\end{itemize}

4. 求解最优税率:
\begin{itemize}
\item 对$W(t)$求导并令其等于零:
\[\frac{\partial W}{\partial t}=0\]
\item 解得最优税率:
\[t^*=-A+\frac{B}{3}\]
\end{itemize}

5. 结果分析:
\begin{itemize}
\item 在给定约束条件$A<3B$和$B<3A$下:
    \begin{itemize}
    \item 最优税率$t^*=-A+\frac{B}{3}$必定为负值
    \item 这意味着最优政策一定是补贴
    \end{itemize}
\end{itemize}

6. 经济直觉:
\begin{itemize}
\item 在给定的约束条件下:
    \begin{itemize}
    \item 国内外市场需求参数的差异不会太大($\frac{1}{3}<\frac{B}{A}<3$)
    \item 补贴可以降低价格,增加消费者剩余
    \item 补贴成本由企业的额外利润部分抵消
    \end{itemize}
\item 约束条件$A<3B$和$B<3A$确保了解的合理性:
    \begin{itemize}
    \item 保证了两个市场都有正的销量
    \item 避免了极端的补贴政策
    \end{itemize}
\end{itemize}

这个结果表明,在国内外市场需求差异不太大的情况下,最优贸易政策是对垄断企业进行适度补贴,以提高本国福利。补贴可以降低价格,增加消费者剩余,同时企业的额外利润可以部分抵消补贴成本。

\section*{第十题(4分)}
Peltzman(1976)关于政府规制的政治经济学模型认为,最可能被规制的市场要么是那些竞争十分激烈的市场,要么是垄断程度较高的产业。请解释为什么。

\noindent\textbf{答:}Peltzman的模型从政治经济学角度揭示了政府规制的两极化特征。在竞争激烈的市场中,大量企业参与竞争导致利润微薄,这促使企业寻求通过规制来限制竞争,提高行业整体利润水平。由于企业有强烈的组织动机,且规制成本可以在行业内分摊,同时消费者反对的组织成本较高,这种规制在政治上是可行的。

相反,在垄断程度高的产业中,企业具有显著的市场势力,价格远高于边际成本。这种情况下,消费者要求限制垄断力量的呼声强烈,希望通过规制降低价格,增加产量。由于垄断问题容易引起公众关注,且规制收益集中于消费者,政治家可以通过支持规制获得选民支持。

而对于中间市场结构,由于企业和消费者的利益诉求都不够强烈,规制的政治收益不够明显,难以形成有效的政治压力,因此较少被规制。这种规制模式反映了政府规制不仅受经济效率的影响,更多地受到政治过程和利益集团博弈的影响。

\section*{第十一题(6分)}
社会资源有效配置的特征是:以最低的机会成本、生产出社会最优的产品种类和数量、并配置给最需要的消费者。

\begin{enumerate}
\item 在一个价格过高的垄断市场,社会资源配置是否有效?为什么?(3分)
\item 在一个没有政府规制的"自然垄断"行业,资源配置是否有效?为什么?(3分)
\end{enumerate}

\noindent\textbf{答:}

(a) 在价格过高的垄断市场中,社会资源配置是无效的。垄断企业为了最大化利润,选择边际收入等于边际成本的产量水平,这导致市场价格高于边际成本。过高的价格使得一些愿意支付高于边际成本但低于垄断价格的消费者无法获得产品,产生了无谓损失。这种情况下,资源未能配置给所有边际效用大于边际成本的消费者,造成社会总剩余的损失,资源配置偏离了帕累托最优状态。

(b) 在未受规制的自然垄断行业中,资源配置同样是无效的,但问题更为复杂。自然垄断行业具有显著的规模经济特征,平均成本随产量增加而下降,使得单个企业生产比多个企业更有效率。然而,这种情况下存在一个定价困境:如果价格等于边际成本以实现配置效率,企业会亏损;如果价格等于平均成本以维持企业生存,又会高于边际成本造成配置无效;如果允许垄断定价,问题会更加严重。此外,缺乏竞争压力还可能导致X-效率损失和创新动力不足。这种情况需要通过政府规制,在效率和企业生存之间寻找适当的平衡点。

这两种情况都说明,市场力量的存在会导致资源配置偏离社会最优水平,需要适当的政策干预来改善资源配置效率。但在自然垄断的情况下,问题更为棘手,因为效率和企业生存之间存在根本性的冲突。

\end{document} 