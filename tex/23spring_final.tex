% !TEX program = xelatex
\documentclass[12pt]{article}
\usepackage{ctex}
\usepackage{amsmath}
\usepackage{amssymb}
\usepackage{geometry}
\usepackage{fancyhdr}
\usepackage{titlesec}
\usepackage{xcolor}
\usepackage[shortlabels]{enumitem}

\geometry{a4paper,margin=2.5cm,headheight=15pt}
\linespread{1.3}

% 设置页眉页脚
\pagestyle{fancy}
\fancyhf{}
\fancyhead[L]{Chen Gao}
\fancyhead[R]{2023年春季学期}
\fancyfoot[C]{\thepage}

% 设置section格式
\renewcommand{\thesection}{\chinese{section}}
\titleformat{\section}
{\normalfont\large\bfseries}{第\thesection 题}{1em}{}
\titlespacing{\section}{0pt}{3.5ex plus 1ex minus .2ex}{2.3ex plus .2ex}

% 设置enumerate环境
\setenumerate[1]{label=(\alph*),leftmargin=2em}

\begin{document}

\title{产业组织理论期末考试试题}
\author{Chen Gao\thanks{北京大学国家发展研究院,仅作为本人期末复习自制答案,不保证正确性}}
\date{\today}
\maketitle




\medskip
\noindent\textbf{请回答所有考题,表述务必清楚,准确,简洁。}

\bigskip
\section*{第一题(6分)}
考虑一个有两个产品(1和2)的市场,其中产品1由一个垄断企业提供,产品2由完全竞争的企业提供。如果上述垄断企业从市场上采购产品2,然后将其与产品1进行"条件捆绑"(即消费者从垄断企业购买产品1时,必须同时从该企业购买其所需要的产品2,否则拒绝交易)后销售给消费者。如果消费者不接受垄断企业的条件捆绑方案,仍然可以在市场上单独购买产品2,但无法买到产品1。该垄断企业的"条件捆绑"销售方式是否有利可图?请从直观上给出一个解释。

\noindent\textbf{答:}通过这种方式,垄断企业可以将自己在垄断商品上的市场势力延伸到完全竞争商品上。通过适当降低捆绑组合中产品1的价格到垒断价格之下并提高产品2的价格到完全竞争价格之上,可以达到使消费者购买组合商品的剩余大于在完全竞争市场单独购买产品2的剩余(3),同时垄断企业在产品2上回收的额外利润大于在产品1上的利润损失,使企业有利可图(3)

\section*{第二题(6分)}
某垄断企业提供两个产品。企业可以对两个产品分别定价销售,也可以进行比例捆绑后按一个总价销售。试给出一个数值例子,说明比例捆绑对企业而言可能是有利可图的。

\noindent\textbf{答:}考虑A,B消费者对产品1,2估值如下的情况

$$\begin{array}{c|cc}&1&2\\\hline A&20&30\\[6pt]B&26&14\end{array}$$

分别定价:对于产品1,企业的利润作为定价的函数是

$$\pi(p_1)=\begin{cases}2p_1,\quad p_1\le20\\[2ex]p_1,\quad20<p_1\le26\\[2ex]0,\quad p_1>26\end{cases}$$

对于产品2,企业的利润作为定价的函数是

$$\pi(p_2)=\begin{cases}2p_2,\quad p_2\le14\\[2ex]p_2,\quad14<p_2\le30\\[2ex]0,\quad p_2>30\end{cases}$$

最优定价是$p_1= 20$, $p_2= 30$,相应的利润为70.

捆绑销售:按照1:1捆绑,企业的利润作为定价的函数是

$$\pi(p)=\begin{cases}2p,&p\le40\\[2ex]p,&40<p\le50\\[2ex]0,&p>50\end{cases}$$

最优定价是40,相应利润为80

因此,捆绑销售的利润更高。

\section*{第三题(16分)}
某行业有两个寡头企业,分别记为1和2,他们的总成本函数均为$c(q)=1+9q$。市场需求函数为$P=15-Q$。现假设企业1有一个技术改造机会,能够以15的技改投资将边际成本从9降低至$c<6$的水平。

\begin{enumerate}
\item 请问企业1是否会进行该项技改投资?(8分)
\item 假如企业1进行该投资,对消费者福利有什么影响?(8分)
\end{enumerate}

\noindent\textbf{答:}

a)没有技改时,两企业的利润为

$$\pi_1^1=\pi_2^1=\frac{(15+9-2\times9)^2}{9}-1=3.\quad(2')$$

企业1进行技改后,两企业的利润为

$$\pi_1^2=\frac{(15+9-2c)^2}{9}-1-15>0\quad(2'),\quad\pi_2^2=\frac{(15+c-2\times9)^2}{9}-1<0\quad(2').$$

此时企业2在市场中会承担亏损,企业2选择退出,企业1的利润变为

$$\pi_1^3=\frac{(15-c)^2}{4}-1-15>\frac{17}{4}>3.\quad(2')$$

此时企业1的利润比原先更大,企业1会进行技改

b)在技改前,市场的产量和消费者剩余分别为

$$q_1=q_2=2\quad(2'),\quad CS_1=8\quad(2').$$

技改后,产量和消费者剩余为

$$q=\frac{15-c}{2}>\frac{9}{2}\quad(2'),\quad CS_2=\frac{(15-c)^2}{8}>\frac{81}{8}>8\quad(2').$$

虽然企业1的技改对企业2起到了阻止进人的目的,使市场结构由寡头变为垒断,但是技改同时也显著降低了生产的边际成本,最终市场总产量增加,价格下降,消费者的福利相较于原先增加。

\section*{第四题(16分)}
某行业有一个垄断的上游企业和两个寡头下游企业,每个上游产品可用于生产一个下游产品,上游企业的边际成本均为0,两个下游企业(在购买上游产品之外)的边际成本也均为0。下游企业的产品无差异,市场需求函数为$P(Q)=A-Q$,其中参数$A$足够大。

\begin{enumerate}
\item 假设上游企业首先决定上游产品价格$w$,然后下游企业选择他们的产量。请找出这个市场的均衡总产量、价格、和产业总利润。(8分)
\item 假如上游企业可以通过合约决定下游产品的价格,请找出这个市场的均衡总产量、价格、和产业总利润。(8分)
\end{enumerate}

\noindent\textbf{答:}

a)得知上游企业产品的定价后,两个下游企业分别最大化自已的利润

$$\max_{q_{1}}\:(A-q_{1}-q_{2}-w)q_{1}\\\max_{q_{2}}\:(A-q_{1}-q_{2}-w)q_{2}\quad(2')$$

解得下游企业的产量决定为

$$q_1^*=q_2^*=\frac{A-w}{3}.\quad(2')$$

上游企业在考虑到下游企业的决定后,选择定价最大化自已的利润

$$\max_w\:2w\frac{A-w}{3}\quad(2')$$

最终得到均衡下

$$w^*=\frac{A}{2},\quad p^*=\frac{2A}{3},\quad q^*=q_1^*+q_2^*=\frac{A}{3},\quad\pi^*=\pi_u^*+2\pi_d^*=\frac{A^2}{6}+2\times\frac{A^2}{36}=\frac{2A^2}{9}.$$

b)上游企业一旦决定下游企业的价格,需求量随之确定,下游企业已经没有可以操控的变量。上游企业的利润最大化问题为

$$\max_{p}\:p(A-p)\quad(2')$$

均衡结果为

$$p^*=\frac{A}{2}\quad(2'),\quad q^*=\frac{A}{2}\quad(2'),\quad\pi^*=\frac{A^2}{4}\quad(2').$$

\section*{第五题(16分)}
某耐用品垄断企业生产可使用2期的产品,边际生产成本为$c>0$。消费者在第一期对该耐用品服务的租赁需求为$R(Q)=A-Q$,第二期为$R(Q)=B-Q$,其中$R$为租价,$Q$为需求量,参数$A$和$B$满足$0<A<B$。市场仅存在2期,两期之间的贴现因子为1。企业以出租的方式获得收入。请找出垄断企业在两期的最优产量。

\noindent \textbf{答:}

记企业两期的产量分别是$Q_1,Q_2$,则其利润最大化问题为

$$\max\limits_{Q_1,Q_2}\:(A-Q_1)Q_1+(B-Q_1-Q_2)(Q_1+Q_2)-c(Q_1+Q_2)\quad($$

$$\begin{array}{ccc}\mathrm{s.t.}&Q_1\geq0&(1')\\[2ex]&Q_2\geq0&(1')\end{array}$$

构造

$$\mathcal{L}=(A-Q_1)Q_1+(B-Q_1-Q_2)(Q_1+Q_2)-c(Q_1+Q_2)+\mu_1Q_1+\mu_2Q_2$$

库恩塔克条件为

$$\begin{cases}A+B-c-4Q_1-2Q_2+\mu_1=0\quad(2')\\[2ex]B-c-2Q_1-2Q_2+\mu_2=0\quad(2')\\[2ex]\mu_1Q_1=0\\[2ex]\mu_2Q_2=0\\[2ex]\mu_1,\:\mu_2\geq0\\[2ex]Q_1,\:Q_2\geq0\end{cases}$$

当$\mu_1=0$, $\mu_2=0$时,$c\leq B-A$的条件下有解

$$Q_1^*=\frac{A}{2},\quad Q_2^*=\frac{B-A-c}{2}.$$

当$\mu_1>0$, $\mu_2=0$时,无解。

当$\mu_1=0$, $\mu_2>0$时,$B-A<c\leq A+B$的条件下有解

$$Q_1^*=\frac{A+B-c}{4},\quad Q_2^*=0.$$

当$\mu_1>0$, $\mu_2>0$时,$c>A+B$的条件下有解

$$Q_1^*=0,\quad Q_2^*=0.$$

结合题中参数范围,解为

$$Q_1^*=\frac{A}{2}\quad(2'),\quad Q_2^*=\frac{B-A-c}{2}\quad(2').$$

\section*{第六题(18分)}
某行业的市场需求函数是$P=12-Q$。当前生产技术的边际成本为4。假设有一个独立实验室获得了一项技术,可以将生产的边际成本降低至2。

\begin{enumerate}
\item 如果这个市场上有一个垄断者,并且实验室可以将技术无条件授权给该垄断企业使用,请问实验室最多可以收到多少使用费?(6分)
\item 如果这个市场上有一个垄断者,并且实验室将技术以计件收费的方式授权给该垄断企业使用,请问实验室最多可以收到多少使用费?(6分)
\item 如果这个市场有两个寡头企业,他们进行产量竞争。如果实验室将技术以计件收费的方式授权给两个企业使用,请问实验室最多可以收到多少使用费?(6分)
\end{enumerate}

\noindent \textbf{答:}

a)在边际成本为$c$ 时,垄断企业最优产量与相应利润分别为

$$q^*(c)=\frac{12-c}{2},\quad\pi^*(c)=\frac{(12-c)^2}{4}.$$
(2)

使用费为拥有技术后和拥有技术前垄断企业利润之差

$$f=\pi^*(2)-\pi^*(4)=9.\quad(2')$$

b)假设单件费用为$r$ ,拥有技术时,垄断企业最优产量与相应利润分别为

$$q^*(r)=\frac{10-r}{2},\quad\pi^*(r)=\frac{(10-r)^2}{4}.$$
(2')

对比a)问,只有当单件价格不超过2时,垄断企业才有动机购买这项技术,因此实验室的利润最大化问题为
$$\max_{r\leq2}\:\frac{r(10-r)}{2}\quad(1')$$

解得最优单件价格、垄断企业产量、总使用费分别为

$$r^*=2,\quad q^*=4,\quad f=8.\quad(1')$$

c)在边际成本为$c$ 时,单个寡头企业最优产量与相应利润分别为

$$q^*(c)=\frac{12-c}{3},\quad\pi^*(c)=\frac{(12-c)^2}{9}.\quad(2$$

使用费为拥有技术后和拥有技术前两个寡头企业利润之差

$$f=2\pi^*(2)-2\pi^*(4)=8.\quad(2')$$

d)假设单件费用为$r$ ,拥有技术时,单个寡头企业最优产量与相应利润分别为

$$q^*(r)=\frac{10-r}{3},\quad\pi^*(r)=\frac{(10-r)^2}{9}.\quad(2')$$

对比c)问,只有当单件价格不超过2时,寡头企业才有动机购买这项技术,因此实验室的利润最大化问题为
$$\max_{r\leq2}\:\frac{2r(10-r)}{3}\quad(1')$$

解得最优单件价格、单个寡头企业产量、总使用费分别为

$$r^*=2,\quad q^*=\frac{8}{3},\quad f=\frac{32}{3}.\quad(1')$$

\section*{第七题(14分)}
某产品在全球只有两个寡头生产商,都位于某国。两个企业以零成本生产同质产品,并在国内和国际市场销售。企业之间进行静态产量竞争。假设全球消费者都是同质的,一个代表性消费者对该产品的需求为$P(Q)=12-Q$,$Q\geq0$。国际国内市场规模的比例为$\alpha:1$。假如这个两个企业合并,从而成为全球市场的垄断者。请问从本国利益出发,这个合并是否有利?

\noindent \textbf{答:}

让我们分别计算合并前后的本国福利。

在寡头竞争下,每个企业的最优产量为$q=4$,市场总产量为$Q=8$,市场价格为$P=4$。

每个企业的总利润为$\pi=4\times4\times(1+\alpha)=16(1+\alpha)$,两个企业的总利润为$\Pi=32(1+\alpha)$。

国内消费者剩余为$CS_d=\frac{8^2}{2}=32$。

因此,合并前的本国福利为$W_1=32(1+\alpha)+32=32\alpha+64$。

在垄断情况下,每个市场的产量为$Q_m=6$,价格为$P_m=6$。

垄断企业的总利润为$\pi_m=6\times6\times(1+\alpha)=36(1+\alpha)$。

国内消费者剩余为$CS_m=\frac{6^2}{2}=18$。

因此,合并后的本国福利为$W_2=36(1+\alpha)+18=36\alpha+54$。

本国福利的变化为$\Delta W=W_2-W_1=(36\alpha+54)-(32\alpha+64)=4\alpha-10$。

当$\Delta W>0$时,即$4\alpha-10>0$,解得$\alpha>\frac{5}{2}$。

因此,当国际市场规模是国内市场的2.5倍以上时,合并有利于本国福利。这是因为虽然合并会降低国内消费者剩余,但当国际市场足够大时,企业从国际市场获得的额外垄断利润可以弥补国内消费者剩余的损失。

\section*{第八题(8分)}
一个国家的某产品完全依赖一个外国垄断企业,该产品的边际生产成本为常数。本国对该产品的需求函数为线性(如$Q=A-P$,其中参数$A$足够大)。在完全自由贸易条件下,本国大量进口该产品。请问在不考虑两国间贸易争端的前提下,本国政府是否有动机单方面对该产品的进口进行征税?为什么?

\noindent \textbf{答:}

本国政府有动机对进口产品征税。让我们通过分析来说明原因:

(1) 在征税前:
外国垄断企业面对需求函数$Q=A-P$,其利润最大化问题为:
\[\max_P (P-c)(A-P)\]

一阶条件得到最优价格$P^*=\frac{A+c}{2}$,此时:
\begin{itemize}
\item 产量:$Q^*=\frac{A-c}{2}$
\item 消费者剩余:$CS^*=\frac{(A-c)^2}{8}$
\end{itemize}

(2) 征收单位税率为$t$的关税后:
外国企业的利润最大化问题变为:
\[\max_P (P-c-t)(A-P)\]

一阶条件得到最优价格$P^*(t)=\frac{A+c+t}{2}$,此时:
\begin{itemize}
\item 产量:$Q^*(t)=\frac{A-c-t}{2}$
\item 消费者剩余:$CS(t)=\frac{(A-c-t)^2}{8}$
\item 关税收入:$T(t)=t\cdot Q^*(t)=t\frac{A-c-t}{2}$
\end{itemize}

(3) 本国福利为消费者剩余与关税收入之和:
\[W(t)=CS(t)+T(t)=\frac{(A-c-t)^2}{8}+t\frac{A-c-t}{2}\]

对$t$求导并令其等于0,得到最优关税:
\[t^*=\frac{A-c}{3}\]

这个关税水平是正的,且能使本国福利最大化。这说明本国确实有动机征税。

征税的经济直觉是:通过征税,本国可以从外国垄断企业手中攫取一部分垄断利润。虽然征税会提高价格,降低消费者剩余,但适当水平的关税所带来的收入可以超过消费者剩余的损失,从而提高本国总福利。这就是所谓的"最优关税"理论。

\end{document}
